%!TEX root = memoria_duocode_interfaz.tex
El proyecto ``Desarrollo de un front-end para Duocode'' tiene como objetivo el desarrollo de una aplicación web para el aprendizaje de lenguajes de programación. Permite a los usuarios aprender lenguajes de programación a partir de alguno que ya sepan mediante la superación de distintas lecciones y temas. Al empezar, sólo algunas lecciones de cada tema están disponibles, el resto se van desbloqueando a medida que se van superando las anteriores.

Los temas consisten en una serie de lecciones. Las lecciones se componen de una colección de ejercicios, que se basan en un enunciado en el lenguaje que el usuario conoce y que tendrá que ser traducido al lenguaje que quiere aprender.

Como un fragmento de código en un lenguaje específico puede escribirse de distintas maneras, una parte de la aplicación irá dedicada a los \textbf{candidatos}. Si un usuario falla en la resolución de un ejercicio pero cree que su solución es correcta, tiene la posibilidad de enviar su ejercicio como candidato. Al hacer esto, dicha solución pasa a ser evaluada por otros usuarios, si obtiene muchos votos positivos y un usuario administrador la da por válida, pasa a ser solución correcta de ese ejercicio a partir de ese momento.

Duocode también permite marcar ejercicios como favorito para tenerlos accesibles y poder consultarlos en cualquier momento.

Para acceder como usuario no hay que registrarse en la web ya que incluye un inicio de sesión con \textbf{Facebook} y con \textbf{Google+} para el que lo único que hace falta es darle los permisos de acceso a la información básica del perfil de usuario de la correspondiente red social. Cuenta también con la posibilidad de compartir en Facebook el éxito tras superar una lección.

 

