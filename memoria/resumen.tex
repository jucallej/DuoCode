%!TEX root = memoria_duocode_interfaz.tex

El proyecto ``Desarrollo de un front-end para Duocode'' tiene como objetivo el desarrollo de una aplicación web para el aprendizaje de lenguajes de programación. Esta aplicación permite a los usuarios aprender lenguajes de programación a partir de alguno que ya sepan mediante la superación de distintas lecciones y temas. Al empezar, solo algunas lecciones de cada tema están disponibles; el resto se van desbloqueando a medida que se van superando las anteriores.

Para conseguir una estructura clara, los temas en DuoCode consisten en una serie de lecciones. Asimismo, las lecciones se componen de una colección de ejercicios, que se basan en un enunciado en el lenguaje que el usuario conoce y que tendrá que ser traducido al lenguaje que quiere aprender.

A medida que el usuario va resolviendo los ejercicios propuestos, su puntuación va aumentando. Además, el usuario dispone de vidas, las cuales se restarán cuando la respuesta no sea correcta. Esto hace que la aplicación cuente con un tipo de aprendizaje más entretenido, haciéndolo ver como un juego. Otra herramienta de la que dispone Duocode es que permite marcar ejercicios como favorito para tenerlos accesibles y poder consultarlos en cualquier momento.

Además, como un fragmento de código en un lenguaje específico puede escribirse de distintas maneras, una parte de la aplicación está dedicada a los \textbf{candidatos}. Si un usuario falla en la resolución de un ejercicio pero cree que su solución es correcta, tiene la posibilidad de enviar su ejercicio como candidato. Al hacer esto, dicha solución pasa a ser evaluada por otros usuarios. Si obtiene los suficientes votos positivos y un usuario administrador la da por válida, pasa a ser solución correcta de ese ejercicio a partir de ese momento; por el contrario, si se vota negativamente esta solución se descartará y no podrá volver a proponerse.

Para acceder como usuario no hay que registrarse en la web ya que incluye un inicio de sesión con \textbf{Facebook} y con \textbf{Google+}. Por ello, lo único necesario para utilizar la aplicación es
darle los permisos de acceso a la información básica del perfil de usuario de la correspondiente red social. Asimismo, DuoCode cuenta también con la posibilidad de compartir en Facebook el éxito tras superar una lección.

Por último, DuoCode es una herramienta útil no solo para los estudiantes, también para los docentes, pues su base de datos guarda incluso información sobre los envíos realizados por los estudiantes. De esta manera se puede hacer un seguimiento de la evolución de los usuarios y detectar aspectos problemáticos que se deban reforzar en el aula.

\textbf{Palabras clave:} Web, REST, AngularJS, base de datos, Bootstrap, aprendizaje, lenguajes de programación, traducción, candidato, favorito.