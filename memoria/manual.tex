En este apartado se detallarán los pasos a seguir para tener instalado \textbf{DuoCode} en el sistema, con el fin de poder trabajar directamente con los ficheros fuentes y extender el proyecto. El software necesario para trabajar con el proyecto se especifica a continuación.

\begin{itemize}
\item \textbf{Requisitos generales}
\begin{itemize}
\item Java 7
\item Git
\item Navegador Web - Google Chrome
\end{itemize}
\end{itemize}
\begin{itemize}
\item \textbf{Requisitos para el Servicio Web}
\begin{itemize}
\item Xampp, Wampp o similar
\item Tomcat 8.0
\item NetBeans 8.0
\end{itemize}
\end{itemize}
\begin{itemize}
\item \textbf{Requisitos para el Front-End}

\begin{itemize}
\item Soporte SSL en Xampp y Tomcat
\end{itemize}
\end{itemize}
\begin{itemize}
\item \textbf{Requisitos para la Aplicación móvil}
\begin{itemize}
\item Node.Js 0.10.X
\item Phonegap 5.0.0
\item Eclipse Luna con plugin para Android
\end{itemize}
\end{itemize}

\subsection{Instalación paso a paso}
\begin{itemize}
\item \textbf{Java}

En primer lugar necesitaremos tener instalada una versión de Java en nuestro sistema. La forma más sencilla es acceder a la web http://www.java.com/es/ y descargar la versión adecuada para nuestro sistema operativo.

Si lo prefieres puedes usar la línea de comandos:
{\codesize
\begin{verbatim}
$ sudo apt-get install openjdk-7-jdk openjdk-7-jre
\end{verbatim}
}


\item \textbf{Git}

Para poder descargar todos los ficheros fuentes del proyecto es necesario tener un cliente Git instalado y configurado.

Una opción es acceder a la web http://git-scm.com/downloads/guis y elegir el cliente que queramos para nuestro sistema operativo. 

Si lo prefieres puedes usar la línea de comandos:

{\codesize
\begin{verbatim}
$ sudo apt-get install git
\end{verbatim}
}

Una vez instalado git podremos clonar el proyecto DuoCode y obtener una copia en nuestro sistema. Si trabajas con un cliente gráfico simplemente tendrás que pinchar en el botón 'clone' y escribir la dirección del repositorio:

{\codesize
\begin{verbatim}
https://github.com/jucallej/DuoCode.git
\end{verbatim}
}

También puedes clonar el proyecto directamente desde la línea de comandos:
{\codesize
\begin{verbatim}
$ git clone https://github.com/jucallej/DuoCode.git
\end{verbatim}
}


\item \textbf{Google Chrome}

Cualquier navegador sería válido pero Chrome cuenta con un plugin (Advance Rest Client) para hacer pruebas con el servicio web Rest bastante intuitivo. Para instalar el navegador accedemos a la web de Chrome https://www.google.es/chrome/ y pinchamos en el botón de descarga.

Si lo prefieres también puedes usar la línea de comandos:
{\codesize
\begin{verbatim}
$ sudo apt-get install google-chrome-stable
\end{verbatim}
}



\item \textbf{XAMPP - WAMP - MAMP o similar}

Será necesario tener en local instalado un servidor como cualquiera de los nombrados anteriormente. Nos proporcionan una base de datos MySQL y un server Apache.
Para instalarlo solo hay que acceder a la web "https://www.apachefriends.org/index.html" (en el caso de querer instalar XAMPP) y descargar la versión para nuestro sistema operativo.


Una vez instalado y funcionando accedemos a http://localhost/phpmyadmin  para importar la base de datos.
Creamos una Base de datos nueva con el nombre 'Duocode', la seleccionamos e importamos el archivo 'Duocode.sql' para que nos cree las tablas y cargue la información del proyecto.


En la carpeta 'htdocs' que se encuentra dentro de la carpeta del servidor XAMPP es donde tendremos que poner la parte del front-end.
Copiamos la carpeta 'duocode' que contiene el index.html y todos los scripts y la pegamos en 'htdocs'.
Podemos probar el funcionamiento accediendo a la dirección http://localhost/duocode



\item \textbf{Tomcat}

Para el servicio web REST necesitaremos tener instalado Tomcat 8.0 o superior.

Accedemos a la web http://tomcat.apache.org/download-80.cgi (para la versión 8.0), descargamos la última versión para nuestro sistema operativo y lo descomprimimos.



\item \textbf{NetBeans}

El IDE que se ha usado para desarrollar el proyecto es NetBeans 8.0 o superior por su facilidad para integrar el control de versiones Git y los servidores, se puede descargar desde la web https://netbeans.org/downloads/ la versión para nuestro sistema operativo.

Una vez instalado importamos el proyecto descargado desde el Git y seleccionamos el servidor con el que queremos que funcione, en nuestro caso será el Tomcat descargado anteriormente.

Las librerías necesarias se importarán de manera automática una vez que hayamos cargado el proyecto en NetBeans.

\item \textbf{Certificados SSL}

Para que tanto el front-end como el servicio web funcionen con https es necesario activar SSL en los servidores Apache del XAMPP y el Tomcat descargado.

Tomcat lo podemos configurar gracias al archivo server.xml encontrado en /apache-tomcat-8.0.21/conf/server.xml
Lo abrimos y dentro del elemento:
{\codesize
\begin{verbatim}
 < Service name = 'Catalina' >  
\end{verbatim}
}
pegamos estas líneas de código (cambiando la ruta del proyecto):

{\codesize
\begin{verbatim}
<Connector port="8443" protocol="org.apache.coyote.http11.Http11NioProtocol"
               maxThreads="150" SSLEnabled="true" scheme="https" secure="true"
               clientAuth="false" sslProtocol="TLS"
           keystoreType="PKCS12"
           keystoreFile="{Ruta al repositorio}\DuoCode\duocode\certificados\mycert.p12" keystorePass="contraseña"/>
\end{verbatim}
}

Para configurar el server Apache que nos proporciona XAMPP primero tendremos que poner los archivos "mars-server.crt, mars-server.key, my-ca.crt" en la siguiente ruta:

'/opt/lampp/etc/ssl.crt/' los ficheros con la extensión .crt

'/opt/lampp/etc/ssl.key/' el fichero con la extensión .key

Una vez que tenemos los certificados y la clave en las rutas adecuadas editamos el fichero httpd-ssl.conf. Dependiendo de la versión puede tener un aspecto u otro, en nuestro caso hemos definido un nuevo VirtualHost con la siguiente información. En el caso de trabajar bajo un sistema Linux se puede copiar y pegar las siguientes líneas en el archivo:

{\codesize
\begin{verbatim}
<VirtualHost _default_:443>

DocumentRoot "/opt/lampp/htdocs"
ServerName localhost:443
ServerAdmin you@example.com
ErrorLog "/opt/lampp/logs/error_log"
TransferLog "/opt/lampp/logs/access_log"

SSLEngine on

SSLCertificateFile "/opt/lampp/etc/ssl.crt/mars-server.crt"
SSLCertificateKeyFile "/opt/lampp/etc/ssl.key/mars-server.key"
SSLCertificateChainFile "/opt/lampp/etc/ssl.crt/my-ca.crt"

SSLCACertificateFile "/opt/lampp/etc/ssl.crt/my-ca.crt"

<FilesMatch "\.(cgi|shtml|phtml|php)$">
    SSLOptions +StdEnvVars
</FilesMatch>
<Directory "/opt/lampp/cgi-bin">
    SSLOptions +StdEnvVars
</Directory>

BrowserMatch "MSIE [2-5]" \
         nokeepalive ssl-unclean-shutdown \
         downgrade-1.0 force-response-1.0

CustomLog "/opt/lampp/logs/ssl_request_log" \
          "%t %h %{SSL_PROTOCOL}x %{SSL_CIPHER}x \"%r\" %b"

<Directory "/opt/lampp/htdocs">
        Options Indexes
        AllowOverride None
        Allow from from all
        Order allow,deny
</Directory>

</VirtualHost>
\end{verbatim}
}

Si hemos seguido correctamente las instrucciones ya nos debería permitir acceder a la dirección https://localhost/duocode aunque nos saldrá un mensaje de que el certificado no está verificado (es un certificado que hemos creado nosotros) así que lo añadimos como excepción y ya tendríamos DuoCode instalado.


\item \textbf{Node.js}

Para poder trabajar con PhoneGap - Cordova será necesario tener instalado Node.js en nuestro equipo. Podemos hacerlo descargándolo desde la web 'https://nodejs.org/' o directamente desde un terminal:

{\codesize
\begin{verbatim}
$ sudo add-apt-repository ppa:chris-lea/node.js
$ sudo apt-get update
$ sudo apt-get install nodejs
\end{verbatim}
}

Es necesario que la versión sea 0.8+, podemos comprobarlo tecleando desde el terminal: 

{\codesize
\begin{verbatim}
$ node -v
\end{verbatim}
}

\item \textbf{PhoneGap}

PhoneGap nos servirá para crear una app móvil desde el HTML5, CSS y JS de nuestro proyecto.

Para instalarlo podemos descargarlo desde la web 'http://phonegap.com' o directamente desde un terminal:

{\codesize
\begin{verbatim}
$ npm install -g phonegap
\end{verbatim}
}

\item \textbf{Eclipse y Android SDK}

Gracias al plugin de Android podremos usar Eclipse como IDE para probar la app móvil de DuoCode.

Instalaremos la última versión de Eclipse descargándolo desde la web 'http://eclipse.org'

Una vez descargado lo abrimos y vamos a añadir el plugin para Android.
Pinchamos en \textbf{help - Install New Software } y añadimos la siguiente URL 'https://dl-ssl.google.com/android/eclipse/' y le damos a \textbf{next} hasta el final.

Reiniciamos Eclipse y tendremos que especificar la dirección del \textbf{Android SDK} que acabamos de descargar para que se actualice y tener Eclipse listo.

Podemos importar la carpeta del proyecto que encontramos en DuoCode y probarlo con el emulador.

\end{itemize}

