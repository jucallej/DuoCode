En este apartado se detallan los pasos a seguir para tener instalado \textbf{DuoCode} en un sistema \textit{Linux}, con el fin de poder trabajar directamente con los ficheros fuentes y extender el proyecto. La dirección donde podemos descargar todo el código es \url{https://github.com/jucallej/DuoCode.git}.

El software necesario para trabajar con el proyecto se especifica a continuación.

\begin{itemize}
\item \textbf{Requisitos generales}
\begin{itemize}
\item Java 7.
\item Git.
\item Navegador Web - Google Chrome.
\end{itemize}
\end{itemize}
\begin{itemize}
\item \textbf{Requisitos para el Servicio Web}
\begin{itemize}
\item Xampp.
\item Tomcat 8.0.
\item NetBeans 8.0.
\end{itemize}
\end{itemize}
\begin{itemize}
\item \textbf{Requisitos para el Front-End}

\begin{itemize}
\item Soporte SSL en Xampp y Tomcat.
\end{itemize}
\end{itemize}
\begin{itemize}
\item \textbf{Requisitos para la Aplicación móvil}
\begin{itemize}
\item Node.Js 0.10.X.
\item Phonegap 5.0.0.
\item Eclipse Luna con plugin para Android.
\end{itemize}
\end{itemize}

\subsection{Instalación de desarrollador}

Es importante seguir el orden de instalación que aparece a continuación para no tener problemas de dependencias entre programas. 

\begin{itemize}
\item \textbf{Java}

En primer lugar necesitamos tener instalada una versión de Java. La forma más sencilla es acceder a la web \url{http://www.java.com/es/} y descargar la última versión.

Es posible usar el siguiente comando en el termina:
{\codesize
\begin{verbatim}
$ sudo apt-get install openjdk-7-jdk openjdk-7-jre
\end{verbatim}
}


\item \textbf{Git}

Para poder descargar todos los ficheros fuentes del proyecto es necesario tener un cliente Git instalado y configurado.

Una opción es acceder a la web \url{http://git-scm.com/downloads/guis} y elegir el cliente que queramos. 

Es posible usar el siguiente comando en el termina:

{\codesize
\begin{verbatim}
$ sudo apt-get install git
\end{verbatim}
}

Una vez instalado git podemos clonar el proyecto \textbf{DuoCode} y obtener una copia en nuestro sistema. Si trabajas con un cliente gráfico simplemente hay que pinchar en el botón \textit{clone} y escribir la dirección del repositorio \url{https://github.com/jucallej/DuoCode.git}.

También podemos clonar el proyecto directamente desde la línea de comandos:
{\codesize
\begin{verbatim}
$ git clone https://github.com/jucallej/DuoCode.git
\end{verbatim}
}


\item \textbf{Google Chrome}

Cualquier navegador es válido pero Chrome cuenta con un plugin (Advance Rest Client) que es necesario para hacer pruebas con el servicio REST. Para instalar el navegador accedemos a la web de Chrome \url{https://www.google.es/chrome/} y pinchamos en el botón de descarga.

Es posible usar el siguiente comando en el termina:
{\codesize
\begin{verbatim}
$ sudo apt-get install google-chrome-stable
\end{verbatim}
}



\item \textbf{XAMPP}

Es necesario tener instalado un servidor local. XAMPP nos proporciona una base de datos MySQL y un servidor Apache.
Para instalarlo solo hay que acceder a la web \url{https://www.apachefriends.org/index.html} y descargar la última versión disponible.


Una vez instalado y funcionando accedemos a \url{http://localhost/phpmyadmin} para importar la Base de datos.
Creamos una Base de datos nueva con el nombre `Duocode', la seleccionamos e importamos el archivo 'Duocode.sql' para que nos cree las tablas y cargue la información del proyecto.


En la carpeta \textit{`htdocs'}, que se encuentra dentro de la carpeta del servidor XAMPP, es donde tenemos que poner la parte del front-end.
Copiamos la carpeta \textit{`duocode'} que contiene el \textit{`index.html'} y todos los scripts y la pegamos en \textit{`htdocs'}.
Podemos probar el funcionamiento accediendo a la dirección \url{`http://localhost/duocode'}.



\item \textbf{Tomcat}

Para el servicio web REST necesitamos tener instalado Tomcat 8.0 o superior. Para ello, accedemos a la web \url{http://tomcat.apache.org/download-80.cgi} (para la versión 8.0), descargamos la última versión para nuestro sistema operativo y lo descomprimimos.



\item \textbf{NetBeans}

El IDE que se ha usado para desarrollar el proyecto es NetBeans 8.0. Nos permite integrar el sistema de control de versiones \textit{Git} y los servidores. Se puede descargar desde la web \url{https://netbeans.org/downloads/}.

Una vez instalado importamos el proyecto descargado desde el Git y seleccionamos el servidor con el que queremos que funcione, en nuestro caso será el Tomcat descargado anteriormente.

Las bibliotecas necesarias se importan de manera automática una vez que hemos cargado el proyecto en NetBeans.

\item \textbf{Certificados SSL}

Para que tanto el front-end como el servicio web funcionen con \textit{https} es necesario activar SSL en los servidores \textit{Tomcat} y \textit{Apache} del XAMPP.

Hemos creado nuestros propios certificados SSL y se pueden encontrar en la carpeta \textit{duocode/certificados}. Ña contraseña es \textit{complutense}.

Tomcat lo podemos configurar gracias al archivo \textit{server.xml} encontrado en /apache-tomcat-8.0.21/conf/server.xml.
Lo abrimos y dentro del elemento:
{\codesize
\begin{verbatim}
 < Service name = 'Catalina' >  
\end{verbatim}
}
pegamos estas líneas de código (cambiando la ruta del proyecto):

{\codesize
\begin{verbatim}
<Connector port="8443" protocol="org.apache.coyote.http11.Http11NioProtocol"
               maxThreads="150" SSLEnabled="true" scheme="https" secure="true"
               clientAuth="false" sslProtocol="TLS"
           keystoreType="PKCS12"
           keystoreFile="{Ruta al repositorio}\DuoCode\duocode\certificados\mycert.p12" keystorePass="contraseña"/>
\end{verbatim}
}

Para configurar el server Apache que nos proporciona XAMPP primero tenemos que poner los archivos \textit{`mars-server.crt', `mars-server.key'} en la siguiente ruta:

\begin{itemize}


\item \textit{'/opt/lampp/etc/ssl.crt/'} los ficheros con la extensión .crt.

\item \textit{'/opt/lampp/etc/ssl.key/'} el fichero con la extensión .key.

\end{itemize}

Una vez tenemos los certificados y la clave en las rutas adecuadas editamos el fichero \textit{httpd-ssl.conf}. Dependiendo de la versión puede tener un aspecto u otro, en nuestro caso hemos definido un nuevo VirtualHost con la siguiente configuración.

{\codesize
\begin{verbatim}
<VirtualHost _default_:443>

DocumentRoot "/opt/lampp/htdocs"
ServerName localhost:443
ServerAdmin you@example.com
ErrorLog "/opt/lampp/logs/error_log"
TransferLog "/opt/lampp/logs/access_log"

SSLEngine on

SSLCertificateFile "/opt/lampp/etc/ssl.crt/mars-server.crt"
SSLCertificateKeyFile "/opt/lampp/etc/ssl.key/mars-server.key"
SSLCertificateChainFile "/opt/lampp/etc/ssl.crt/my-ca.crt"

SSLCACertificateFile "/opt/lampp/etc/ssl.crt/my-ca.crt"

<FilesMatch "\.(cgi|shtml|phtml|php)$">
    SSLOptions +StdEnvVars
</FilesMatch>
<Directory "/opt/lampp/cgi-bin">
    SSLOptions +StdEnvVars
</Directory>

BrowserMatch "MSIE [2-5]" \
         nokeepalive ssl-unclean-shutdown \
         downgrade-1.0 force-response-1.0

CustomLog "/opt/lampp/logs/ssl_request_log" \
          "%t %h %{SSL_PROTOCOL}x %{SSL_CIPHER}x \"%r\" %b"

<Directory "/opt/lampp/htdocs">
        Options Indexes
        AllowOverride None
        Allow from from all
        Order allow,deny
</Directory>

</VirtualHost>
\end{verbatim}
}

Una vez configurado podemos acceder a la dirección \url{https://localhost/duocode}, aunque nos saldrá un mensaje indicando que el certificado no está verificado (es un certificado que hemos creado nosotros) así que lo añadimos como excepción y ya tenemos \textbf{DuoCode} instalado.


\item \textbf{Node.js}

Para poder trabajar con PhoneGap - Cordova es necesario tener instalado Node.js en nuestro equipo. Podemos hacerlo descargándolo desde la web \url{`https://nodejs.org/'} o directamente desde un terminal:

{\codesize
\begin{verbatim}
$ sudo add-apt-repository ppa:chris-lea/node.js
$ sudo apt-get update
$ sudo apt-get install nodejs
\end{verbatim}
}

Es necesario que la versión sea 0.8+. Podemos comprobarlo tecleando desde el terminal: 

{\codesize
\begin{verbatim}
$ node -v
\end{verbatim}
}

\item \textbf{PhoneGap}

PhoneGap nos sirve para crear una app móvil desde el HTML5, CSS y JS de nuestro proyecto. Para instalarlo podemos descargarlo desde la web \url{http://phonegap.com} o directamente desde un terminal:

{\codesize
\begin{verbatim}
$ npm install -g phonegap
\end{verbatim}
}

\item \textbf{Eclipse y Android SDK}

Gracias al plugin de Android podemos usar Eclipse como IDE para probar la app móvil de DuoCode. Instalamos la última versión de Eclipse descargándolo desde la web \url{http://eclipse.org}.

Una vez descargado lo abrimos y vamos a añadir el plugin para Android.
Pinchamos en \textbf{help - Install New Software} y añadimos la siguiente URL \url{https://dl-ssl.google.com/android/eclipse/} y seleccionamos \textbf{next} hasta completar la instalación.

Reiniciamos Eclipse y tenemos que especificar la dirección del \textbf{Android SDK} que acabamos de descargar para que se actualice y tener Eclipse listo.

Podemos importar la carpeta del proyecto que encontramos en DuoCode y probarlo con el emulador seleccionando la carpeta del proyecto y pulsando sobre \textit{`Run'}.

\end{itemize}

\subsection{Desplegado modo desarrollador}

Finalmente, una vez instalado y configurado todo, podemos desplegar y trabajar sobre el proyecto local.

Tenemos que seguir una serie de pasos que se detallan a continuación:

\begin{itemize}

\item \textbf{Arranque de XAMPP}

Para arrancar el servidor Apache y tener acceso a la Base de datos es necesario que \textit{XAMPP} esté funcionando. Para ello abrimos un terminal, nos posicionamos en la carpeta donde se haya instalado \textit{`/opt/lampp/'} y escribimos el siguiente comando:
{\codesize
\begin{verbatim}
$ ./xampp start
\end{verbatim}
}
Es posible que se quede parado mientras intenta arrancar el servidor Apache porque necesite la contraseña del certificado. Si esto ocurre escribimos en el terminal \textit{`complutense'} y pulsamos Enter.



\item \textbf{Despliegue del servicio REST}

Para desplegar el servicio REST abrimos Netbeans y seleccionamos el proyecto DuoCode que importamos durante la instalación. Accedemos a \textit{Run} - \textit{Build} y después pulsamos con el botón derecho del ratón sobre la carpeta del proyecto y seleccionamos \textit{Deploy}.

Con estos pasos conseguimos que el servidor Tomcat se inicie y se despliegue el servicio REST.

\item \textbf{Acceso a DuoCode}

Si no se ha producido ningún problema durante la instalación y el despliegue podemos acceder a la web \url{https://localhost/duocode} y probar la aplicación.

\end{itemize}