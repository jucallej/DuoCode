%!TEX root = memoria_duocode_interfaz.tex

Todos nos hemos ayudado y colaborado durante este proyecto pero en esta sección explicaré las tareas que yo, José Carlos Valera Villalba, específicamente he realizado.

\begin{itemize}
\item
Cuando investigamos los diferentes enfoques para realizar servicios web, me encargué de buscar información sobre REST, vi sus ventajas y desventajas frente a \textit{SOAP} y \textit{RCP}.

\item
En la reunión para decidir qué tipo de servicio web implementaríamos compartí la información encontrada sobre REST y mi punto de vista. Discutimos los pros y los contras de cada tecnología y al final decidimos que REST era lo que más se ajustaba a nuestras necesidades.

\item
Hice un ejemplo simple de servicio web REST usando el IDE \textit{Netbeans} y el \emph{framework} \textit{Jersey}. Consistía en devolver el factorial de un número, pasándole el número en la cabecera de la petición \textit{HTTP} y en la \textit{query}. El objetivo era poder comparar distintos \textit{IDEs} y  \emph{frameworks} con los que realizar el servicio web. \textit{NetBeans} era una buena opción por su integración de \textit{Git} y toda la documentación online que hay para desarrollar servicios web usando este entorno.

\item
Hicimos una reunión para decidir el \emph{framework} y el IDE que íbamos a usar y yo fui el organizador. Al final decidimos usar \textit{Jersey} como framework y \textit{Netbeans} como IDE.

\item
Me encargué de crear el proyecto, con la estructura básica de todas las clases que necesitaríamos y lo subí a github.

\item
Descargué el framework \textit{Jersey} y lo integré en el proyecto. Completé las clases creadas con la sintaxis de \textit{Jersey} y comprobé su correcto funcionamiento con datos sencillos usando `Advanced Rest Client', una aplicación para testear servicios web tipo REST.

\item
Creé varias clases que representan el modelo de la aplicación, sus correspondientes `\texttt{mappers}' para acceder a la base de datos explicada en la sección~\ref{sec:accesoBD} y las funciones para los distintos verbos de los recursos REST (GET, POST, PUT y DELETE). En concreto implementé:

\begin{itemize}
\item
Todo lo relacionado con la gestión de los favoritos. La implementación del `\texttt{mapper}' necesario y los métodos del recurso REST.

\item
Todo lo relacionado con las lecciones. La implementación del `\texttt{mapper}' necesario y los métodos del recurso REST.

\item
Todo lo relacionado con los temas. La implementación del `\texttt{mapper}' necesario y los métodos del recurso REST.
\end{itemize}
    
\item
Implementé la vista de las lecciones en HTML. Desarrollé el controlador adecuado en \textit{AngularJS} y las funciones necesarias para la gesión de las lecciones bloqueadas de cada usuario.

\item
Implementé la vista de los temas en HTML. Desarrollé el controlador adecuado en \textit{AngularJS} y la función necesaria para que los temas apareciesen ordenados correctamente.

\item
Implementé la vista de los ejercicios en HTML cuando la respuesta del usuario ha sido correcta e incorrecta. Desarrollé en el controlador adecuado en AngularJS las funciones necesarias para mostrar en la vista el resultado de la corrección del ejercicio.

\item
Implementé la vista de los ejercicios cuando te has quedado sin vidas. En el controlador de ejercicios creé las funciones necesarias para llevar la cuenta de las vidas y que cuando se quedase a cero notificase al usuario.

\item
Me encargué de configurar los servidores (\textit{TomCat} para el servicio web y el Apache que viene en \textit{XAMPP}) para que usasen el protocolo HTTPS y proteger el tráfico de datos.

\item
Implementé las peticiones HTTP al servicio web para obtener la lista de lenguajes disponibles en nuestro sistema.

\item
Implementé las peticiones HTTP al servicio web para obtener todos los temas disponibles en nuestro sistema.

\item
Me encargué de que ciertas partes de la aplicación solo fuesen visibles cuando el usuario está logueado. Por ejemplo `compartir en Facebook' solo se mostraría cuando el usuario se ha identificado usando su cuenta de Facebook. Para ello usé la clase \texttt{ng-hide} de AngularJS y desarrollé las funciones necesarias para gestionar la visibilidad en los controladores.

\item
Me encargué del desarrollo de la aplicación móvil. Usé \textit{PhoneGap} y creé la aplicación Android usando la clase `WebView', que lo que hace básicamente es cargar una vista web en el móvil optimizada para su correcta visualización. \textit{PhoneGap} también gestiona los touch events, por lo que el programador no los tiene que tener en cuenta durante el desarrollo.

\item
En la memoria me encargué de escribir la sección sobre el servicio web, donde explico principalmente la arquitectura usada y el framework que elegimos para implementar la API REST.

\item
En la memoria me encargué de escribir la sección del manual de instalación, donde explico todo el software necesario para extender el proyecto y los pasos a seguir para desplegar \textit{DuoCode} en un sistema Linux.

\item 
En la memoria me encargué de escribir las conclusiones obtenidas después de finalizar el desarrollo del proyecto en inglés y español. También redacté la lista de posibles mejoras que se pueden implementar en el futuro.


\end{itemize}