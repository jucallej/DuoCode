%!TEX root = memoria_duocode_interfaz.tex

Hola \textbf{negrita} \emph{cursiva}


Tres

\begin{itemize}
\item
Primera cosa

\item
Segunda cosa
\end{itemize}

\begin{enumerate}
\item

\item
\end{enumerate}

\begin{description}
\item[Cosa 1.] Esto que vamos a comentar ahora es algo muy importante
porque blabla Esto que vamos a comentar ahora es algo muy importante
porque blabla

\item[Cosa2]
\end{description}

\begin{example}
\end{example}

$
\begin{array}{|c|c|c|}
\hline
\mathtt{\mathrm{hola}} & hola & hola\\
\hline
adios & adios & adios\\
\hline
\end{array}
$

\section{Hola}

\subsection{Introducción\label{subsec:intro}}

Para la sección~\ref{subsec:intro} usaremos el libro \cite{ejerciciosEDAT}

{\codesize
\lstset{language=Java}
\begin{lstlisting}[frame=single]
if (condicion){
	int x = 0;
}
else{
	int x = 1;
}
\end{lstlisting}
}