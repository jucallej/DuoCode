%!TEX root = memoria_duocode_interfaz.tex
En este proyecto hemos realizado un servicio Web REST y un \emph{front-end} que lo usa, para crear un proyecto de aprendizaje colaborativo de lenguajes de programación en base a otros que ya sabes llamado DuoCode. Empezó con el objetivo de permitir a los alumnos que lo desarrollan profundizar en tecnologías y paradigmas que no se aprenden a lo largo del grado o que se dan como meras referencias. Todo ello sin olvidarnos de los valiosos conocimientos ya afianzados que seguro serán de utilidad. A continuación definiremos los objetivos del proyecto. En otras secciones pasaremos a explicar el estado actual del sector, detallaremos las tecnologías que nos han influenciado y usaremos, explicaremos la realización del servicio Web y del \emph{front-end}, hablaremos de los requisitos y la base de datos que creamos, y tendremos un manual de usuario.

\subsection{Propuestas y objetivos\label{subsec:introduction}}

Desarrollaremos un proyecto colaborativo para aprender lenguajes de programación. Como estará a estar centrado en código de programación, decidimos llamarlo DuoCode, pues aprenderás lenguajes de programación nuevos a partir de otros que ya sabes. 
\vspace{1em}

Las principales características que queríamos que tuviese disponibles el usuario son:

\begin{itemize}
\item
Se podrá seleccionar el idioma de programación que sabes y el que quieres aprender. El sistema te mostrará código en el lenguaje que dominas y te pedirá rellenarlo en el que no.

\item
Habrá una clara organización con código agrupado por temas.

\item
El usuario tendrá una puntuación que irá aumentando según avance. Además, los ejercicios estarán agrupados y tendrán una serie de vidas para completarlos. Esto favorecerá  que el usuario se fidelice con la aplicación.

\item
Cuando el usuario falle un ejercicio pero no esté de acuerdo, podrá proponerlo como solución. Con la ayuda de la comunidad y de los moderadores, estos se podrán incorporar como soluciones en un lenguaje de programación determinado.

\item
Se podrán guardar ejercicios favoritos para poder consultarlos de nuevo.

\item
Habrá una interacción con las redes sociales. El usuario se podrá autenticar utilizando alguna de ellas y compartir sus resultados en esta.

\item
Tendrá un componente de \emph{gamificación} que permitirá al usuario ganar puntos por cada ejercicio realizado.
\end{itemize}

Basándonos en las conclusiones extraídas del apartado anterior, nuestro objetivos para el proyecto son:

\begin{itemize}
\item
Desarollar un sistema Web RESTful para el aprendizaje de la programación. Para ello nos planteamos los siguientes pasos:

\begin{itemize}
\item
Especificar los distintos recursos REST y qué información recibe y produce para cada uno de los métodos HTTP (GET, POST, PUT Y DELETE).

\item
Diseñar una base de datos relacional para soportar el servicio, especificando las distintas tablas y sus relaciones.

\item
Implementar y probar el servicio REST.

\end{itemize}

\item
Desarrollar una (o dos) interfaces gráficas que usen el servicio que acabamos de describir:

\begin{itemize}
\item
Diseñar la interfaz usando HTML/CSS, usando datos de prueba.

\item
Conectar la interfaz con el servicio Web mediante llamadas asíncronas y JavaScript, de manera que use los datos reales proporcionados por el servicio REST. Al uso de esta técnica se le suele llamar AJAX (acrónimo en inglés de  \emph{Asynchronous JavaScript And XML}).

\item
En caso de tener tiempo diseñar la interfaz de la aplicación para Android y realizar su implementación, o adaptar la Web para su visualización en dispositivos móviles.

\end{itemize}

\end{itemize}