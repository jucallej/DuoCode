%!TEX root = memoria_duocode_interfaz.tex

Este proyecto empezó con el objetivo de permitir a los alumnos que lo desarrollan profundizar en tecnologías y paradigmas que no se aprenden a lo largo del grado o que se dan como meras referencias. Todo ello sin olvidarnos de los valiosos conocimientos ya afianzados que seguro serán de utilidad. Nos planteamos realizar una \emph{front-end} para un proyecto de aprendizaje colaborativo de lenguajes de programación en base a otros que ya sabes, llamado DuoCode. A continuación pasaremos a explicar el estado actual del sector, luego detallaremos las tecnologías que nos han influenciado y usaremos, y finalmente definiremos los objetivos del proyecto.

\subsection{Revisión del estado del arte\label{subsec:introduction}}

En la actualidad hay numerosas maneras de aprender online. Para empezar, existen recursos típicos donde el que quiere aprender lee y realiza poca interacción. El líder por antonomasia de esta modalidad es Wikipedia \cite{wiki}, una enciclopedia libre y editada colaborativamente en múltiples idiomas. Ante dudas concretas hay sitios de preguntas y respuestas como Stack Overflow \cite{stack}, donde programadores se ayudan mutuamente en base a cuestiones concretas. Asimismo muchas universidades usan plataformas digitales como Moodle \cite{moodle}, que permiten publicar apuntes de las diferentes asignaturas. También existen cursos online, los llamados MOOC (acrónimo en inglés de Massive Open Online Course) donde puedes aprender a distancia materias. Estos requieren  que se suba material como vídeos con explicaciones detalladas. Suelen estar enfocados a la obtención de algún tipo de diploma o certificación. Requieren un esfuerzo constante y duradero por parte del profesorado, que tiene que actualizar el material académico disponible y corregir los distintos trabajos de los alumnos. 
\vspace{1em}

Otra manera de enfocar la enseñanza, sobre todo cuando quieres ampliar algo sobre la que ya tienes una base, es a base de ejercicios. Durante bastantes años se han usado los \emph{jueces online} o \emph{correctores automáticos}. Hay un análisis general sobre este tema, donde se estudia la viabilidad de un proyecto colaborativo para enseñar lenguajes de programación \cite{pimcdDuoCode14}. En él se explica como normalmente este tipo de sistemas, aplicados a lenguajes de programación, se basan en la ejecución de unos casos de prueba para comprobar la corrección de los programas del usuario. La desventaja de estos sistemas es que exige que los instructores desarrollen previamente estas pruebas, tarea poco grata. 
\vspace{1em}

Saliéndonos del mundo de la programación, hay otros ejemplos, como Duolingo \cite{duolingo} que permiten aprender un idioma a partir de otro que ya sabes previamente. Sus características más relevantes son:

\begin{itemize}
\item
Plantear el aprendizaje como un juego, haciendo que el usuarios gane puntos y experiencia según va avanzando. Además, incluye vidas, que hacen centrar la atención en la tarea presente para no tener que reiniciar el nivel.

\item
Guarda información sobre los fallos del usuario para intentar repetir esas preguntas y que aprenda los concepto de manera definitiva.

\item
Incluye prácticas con tiempo y la posibilidad de certificar el nivel del idioma mediante tests online.
\end{itemize}

Hay otras alternativas como Bussu \cite{bussu}, donde los usuarios hacen simultáneamente de alumnos y profesores e interactúan entre ellos en una red social con el objetivo de aprender otro idioma.

\subsection{Influencias tecnológicas\label{subsec:introduction}}

Antes de definir cómo llevar a la práctica este proyecto, quisimos ver desde alto nivel qué tipo de sistema queríamos desarrollar. Todas las asignaturas que nombraremos a continuación son las correspondientes al plan de estudios de la universidad \cite{plan}. Este plan de estudios sigues las pautas del Acuerdo del Consejo de Universidades publicado en el B.O.E \cite{boe}. 
\vspace{1em}

Durante la carrera hemos aprendido, entre otras cosas, lenguajes de programación que nos permiten hacer software ejecutable en ordenadores de sobremesa, en asignaturas como Fundamentos de la Programación, o Tecnologías de la Programación. Aunque éste podría haber sido un enfoque válido, no era el que queríamos seguir. Nos parece que podríamos hacer algo mucho más rápido de probar y usar por primera vez, sin la necesidad de instalar pesado software adicional.
\vspace{1em}

En otras asignaturas como Software Corporativo aprendimos a montar y configurar un sistemas de gestión de contenidos, también llamados CMS (acrónimo en inglés de Content Management System) y con ello montamos una Web. Este enfoque presenta más ventajas, pues permite a cualquiera que tenga un navegador de Internet acceder a nuestra plataforma. Sin embargo, los CMS actuales no permiten hacer cosas tan concretas y específicas como lo que queríamos hacer. Además, dejaríamos sin usar la mayoría de las características de estos y pensamos que no usaríamos todo el potencial que ofrecen estas plataformas.
\vspace{1em}

También cursamos Aplicaciones Web, donde aprendimos a desarrollar una Web desde el principio. Realizar un desarrollo de este tipo nos parecía sumamente interesante, pues nos permitiría un alto grado de flexibilidad con la manera en que queremos realizar el proyecto, y permitiría a los potenciales usuarios probar y usar la plataforma de manera rápida. Además el uso de llamadas asíncronas nos puede permitir hacer la Web plenamente interactiva y fluida ante las acciones del usuario. Por esto decidimos utilizar HTML, CSS y JavaScript en nuestro proyecto. En esta asignatura también nos apoyábamos en conocimientos adquiridos en otras como Bases de Datos y Ampliación de Base de Datos, donde aprendimos a diseñar e implementar bases de datos relacionales y a consumirlas desde las aplicaciones Su uso permite establecer interconexiones (relaciones) entre los datos (guardados en las distintas tablas), y tiene como principal ventajas evitar duplicidades de los datos, y garantizar la integridad de los datos relacionados entre si. Con todos estos conocimientos podemos hacer un sitio Web interactivo y que guarde una gran variedad de datos. Por todo esto decidimos usar una base de datos de este tipo.
\vspace{1em}

También sentíamos particular inclinación por hacer que el contenido estuviera disponible en dispositivos móviles, bien sea adaptando la Web o a través de una aplicación móvil. Algunos estábamos matriculados en el curso de realización de la asignatura en Programación de Aplicaciones para Dispositivos Móviles, y vimos una oportunidad para poner de manifiesto lo que se podría aprender en esa asignatura.
\vspace{1em}

Por último investigamos la utilidad de implementar un servicio Web, una tecnología que utiliza un conjunto de protocolos y estándares para intercambiar datos entre aplicaciones. Lo más parecido que hemos dado a este enfoque es el uso de \emph{sockets} en Ampliación de Sistemas Operativos y Redes. Su uso permite abstraer y separar la manipulación de los datos de la representación de la interfaz del usuario, pudiendo incluso tener varias interfaces para la misma aplicación. Por todo ello decidimos usar este enfoque. Investigamos varias maneras de implementar el servicio Web:

\begin{itemize}
\item
La llamada a procedimiento remoto (RPC) (del inglés, Remote Procedure Call) es un protocolo que permite a un ordenador ejecutar código de manera remota en otro. Vimos que hay diversas maneras de realizar un RPC, basándose en distintas especificaciones, siendo una de las más populares SOAP.

\item
SOAP es un acrónimo en inglés de Simple Object Access Protocol y es un protocolo que define como diferentes ordenadores pueden comunicarse a base del intercambio de mensajes XML. Sus principales desventajas son que se apoya en un descriptor de los servicios disponibles, que hay que actualizar con cada nueva funcionalidad añadida. Usa XML un lenguaje de marcado que ha perdido fuelle frente al más ligero JSON. Además no todos los lenguajes de programación ofrecen facilidades para el uso de este tipo de servicio Web.

\item
REST es un protocolo cliente/servicor sin estado que usa HTTP y los métodos de este protocolo (GET, POST, PUT, DELETE ...) para consultar y modificar los distintos recurso. Con frecuencia a este tipo de sistemas se les llama RESTful. Permiten mucha libertad a la hora de desarrollarlo, y como para consumir este tipo de servicio Web solo hace falta hacer peticiones HTTP es ampliamente soportado. Además permite ofrecer los datos en JSON. Por esto decidimos usar esta tecnología para nuestro proyecto.

\end{itemize}

\subsection{Propuestas y objetivos\label{subsec:introduction}}

Gracias a la investigación de todo lo expuesto anteriormente, fijamos nuestras ideas. Nos decidimos a hacer un un proyecto colaborativo para aprender lenguajes de programación, similar a Duolingo. Como iba a estar centrado en código de programación, decidimos llamarlo DuoCode, pues aprenderás lenguajes de programación nuevos a partir de otros que ya sabes. 
\vspace{1em}

Las principales características que queríamos que tuviese disponibles el usuario son:

\begin{itemize}
\item
Poder seleccionar el idioma de programación que sabes y el que quieres aprender. El sistema te mostrará código en el lenguaje que dominas y te pedirá rellenarlo en el que no.

\item
Habrá una clara organización con código agrupado por temas.

\item
El usuario tendrá una puntuación que irá aumentando según avance. Además, los ejercicios estarán agrupados y tendrá una serie de vidas para completarlos. Esto favorecerá  que el usuario se fidelice con la aplicación.

\item
Cuando el usuario falle un ejercicio pero no esté de acuerdo, podrá proponerlo como candidato a válido. Con la ayuda de la comunidad y de moderadores, estos se podrán incorporar como soluciones en un lenguaje de programación determinado.

\item
Se podrá guardar ejercicios favoritos para poder consultarlos de nuevo.

\item
Habrá una interacción con las redes sociales. Se podrá hacer login con alguna de ellas y compartir tus resultados en esta.
\end{itemize}

Basándonos en las conclusiones extraídas del apartado anterior, nuestro objetivos para el proyecto son:

\begin{itemize}
\item
Desarollar un sistema Web RESTful para el aprendizaje de la programación. Para ello nos planteamos los siguientes pasos:

\begin{itemize}
\item
Especificar los distintos recursos REST, y que información recibe y produce para cada uno de los métodos HTTP (GET, POST, PUT Y DELETE).

\item
Diseñar una base de datos relacional para soportar el servicio, especificando las distintas tablas y sus relaciones.

\item
Implementar y probar el servicio REST.

\end{itemize}

\item
Desarrollar una (o dos) interfaces gráficas que usen el servicio que acabamos de describir:

\begin{itemize}
\item
Diseñar la interfaz usando HTML/CSS, usando datos de prueba.

\item
Conectar la interfaz con el servicio Web mediante llamadas asíncronas y JavaScript, de manera que use los datos reales proporcionados por el servicio REST. Al uso de esta técnica se le suele llamar AJAX (acrónimo en inglés de Asynchronous JavaScript And XML).

\item
En caso de tener tiempo diseñar la interfaz de la aplicación para Android y realizar su implementación, o adaptar la Web para su visualización en dispositivos móviles.

\end{itemize}

\end{itemize}