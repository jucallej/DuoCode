%!TEX root = memoria_duocode_interfaz.tex

Todos nos hemos ayudado y colaborado durante este proyecto pero en esta sección explicaré las tareas que yo, Julián~F. Calleja da~Silva, específicamente he realizado.

\begin{itemize}
\item
Cuando investigamos los diferentes enfoques para realizar servicios web, me encargué de SOAP, vi sus ventajas y desventajas e incluso hice un pequeño ejemplo para entender mejor su funcionamiento.

\item
Hicimos una reunión para ponernos de acuerdo en cual era el mejor enfoque para el servicio web antes de presentar los resultados a los profesores. Me encargué de organizarla y presidirla.

\item
Hice un ejemplo simple de servicio web REST usando el IDE eclipse y el \emph{framework} Spring. Consistía en hacer el factorial de un número. El objetivo era poder comparar distintos IDEs y distintos \emph{frameworks} con los que poder realizar el servicio web.

\item
Me encargué de revisar las tablas de la base de datos y comprobé que estaban como mínimo normalizadas hasta la forma normal de \emph{Boyce-Codd}.

\item
Aporté la clase AbstractMapper explicada en la sección~\ref{sec:accesoBD} y amplié su funcionalidad, permitiendo que el método \texttt{`insert'} devuelviese un entero con el valor del campo autoincremental de la tabla en caso de que exista y $-1$ en caso contrario. Además añadí todo lo necesario para usar la base de datos desde el proyecto REST.

\item
Creé varias clases que representan el modelo de la aplicación, sus correspondientes \texttt{`mappers'} para acceder a la base de datos descrita en la sección~\ref{sec:accesoBD} y las funciones para los distintos verbos de los recursos REST (GET, POST, PUT y DELETE). En concreto implementé:

\begin{itemize}
\item
Todo lo relacionado con los ejercicios, incluyendo la lista de los enunciados de los distintos lenguajes de programación correspondientes un ejercicio en su GET.

\item
Todo lo relacionado con los enunciados, incluyendo un método en su \texttt{`mapper'} para obtener todos los enunciados de un ejercicio específico.

\item
Todo lo relacionado con los envíos, incluyendo un método en su \texttt{`mapper'} para obtener todos los envíos de un usuario. Además creé la clase \texttt{`Puntuador'}, que encapsula la capacidad de evaluar un fragmento de código en un lenguaje concreto para un ejercicio específico.
\end{itemize}
    
\item
Realicé lo necesario para asegurarnos que el usuario se ha autenticado correctamente y es quien dice ser. Para ello creé un paquete llamado \texttt{`autentificacion'}, que contiene clases que preguntan a los servidores de Google y Facebook por los datos de un usuario (nombre, enlace a su foto de perfil, etc.) y otra clase con métodos estáticos que se encarga de comprobar, una vez el usuario está correctamente autenticado, si existe en nuestra base de datos, y en caso contrario lo crea.

\item
Me encargué de que la visualización de la web se adaptara correctamente a distintos dispositivos, teniendo en cuenta principalmente dispositivos móviles y ordenadores de sobremesa. Para ello usé el sistema de columnas de Bootstrap.

\item
Creé los botones que usamos en la web. Además del texto elegí los iconos más adecuados para la acción que representan. Tanto para el estilo como los iconos usé las clases CSS disponibles en Bootstrap.

\item
Hice la barra que muestra el progreso de los ejercicios completados en una lección. Para ello usé componentes de Bootstrap y los configuré para que tuvieran el color y la apariencia deseados.
% Si se quita ' y los configuré para que tuvieran el color y la apariencia deseados' salen menos de dos páginas.

\item
Hice parte de las vistas que muestran la información del usuario. En concreto me encargué de la imagen de perfil, el nombre, los puntos conseguidos hasta el momento y los enlaces a sus favoritos y sus candidatos.

\item
Busqué una biblioteca JavaScript que resaltara la sintaxis de los distintos lenguajes de programación. Descargué \texttt{`highlight.js'} y la configuré adecuadamente.

\item
Realicé la vista donde los usuarios pueden votar las soluciones propuestas a candidatos. Además integré una barra que muestra gráficamente los votos a favor y en contra de los usuarios.

\item 
Realicé la vista que muestra que no hay más candidatos disponibles para votar y te invita a seguir realizando ejercicios.

\item
Me encargué de realizar las peticiones al servicio web para saber los ejercicios que corresponden a una lección.

\item
Controlé cuándo un usuario ha terminado todos los ejercicios necesarios para completar la lección. En ese caso se muestra una pantalla de felicitación y se le invita a seguir aprendiendo.

\item
Realicé la función que se encarga de marcar un ejercicio como favorito y la asocié al botón correspondiente. Además, hice que este cambiara de color según su estado.

\item
Realicé la función que se encarga de crear un candidato a partir de una solución incorrecta y la asocié al botón correspondiente.

\item
Conecté la vista para votar candidatos con el servicio web e hice que el botón para pasar al siguiente candidato tuviera la funcionalidad deseada.

\item
Realicé la función que se encarga de votar a un candidato, la asocié a los botones correspondientes, hice que al votar se actualizase la barra que muestra los votos a favor y en contra y cambié los colores de los botones para votar positiva o negativamente según la acción del usuario.

\item
Añadí toda la estructura genérica para permitir al usuario autenticarse. Para ello usé la biblioteca HelloJS. Además, me encargué de que cuando el usuario está autenticado se mande en cada petición su identificador y su \emph{token}. Por último hice que el componente que se usa cuando quieres información del usuario devuelva uno vacío con el rol $-1$ cuando no hay un usuario autenticado.


\item
En la memoria me encargué de hacer la introducción (sección~\ref{sec:intro}) y su traducción al inglés (sección~\ref{sec:introEn}).

\item
En la memoria me encargué de escribir sobre el \emph{front-end} (sección~\ref{sec:front}).

\item
Realicé una lista con las tareas más importantes realizadas durante el proyecto y me encargué de organizar una reunión y presidirla. En ella revisamos cada tarea, nos acordamos de quién la realizó y lo apuntamos para poder especificarlo en la memoria.

\end{itemize}