%!TEX root = memoria_duocode_interfaz.tex
En esta sección se exponen todos los requisitos existentes de nuestra aplicación, explicando la función de cada uno de ellos, sus parámetros de entrada y su respuesta.

\subsection{localhost/duocode/rest/temas}
En este apartado se explica cómo hacer un GET para obtener todos los temas y un POST para crear un nuevo tema.
\vspace{1em}

\textbf{REQ01 – GET:} Devuelve la lista de todos los temas existentes en modo URL. No hace falta enviarle ninguna información. 

\begin{itemize}
\item[•] Response:
{\codesize
\begin{verbatim}
{`temas': [`localhost/duocode/rest/temas/1', 
`localhost/duocode/rest/temas/2']}
\end{verbatim}
}
\end{itemize}

\textbf{REQ02 – POST:} Crea un tema nuevo. Por una parte le pasamos por Payload la información del tema que queremos crear (título, descripción y orden en el que queremos que se muestre): 
\begin{itemize}
\item[•]Payload: 
{\codesize
\begin{verbatim}
{`titulo': `Bucles', `descripcion': `En este tema veremos bucles while', 
`orden': `2'} 
\end{verbatim}
}

Por otra parte le enviamos por Header el ID del usuario, el token y el network, que nos sirven para comprobar que el usuario es administrador y tiene permisos para hacer esta operación:
\item[•]Header:
{\codesize
\begin{verbatim}
{`idUsuario': `2', `token': `token', `network': `network'}
\end{verbatim}
}

La respuesta que obtenemos está conformada por ``error'' e ``id''. Si la respuesta en el campo ``error'' es afirmativa significa que algo ha fallado y no ha podido terminar la operación correctamente, en este caso el id será ``-1'', ya que no ha sido asignado ninguno; si es negativa todo ha ido correctamente y el id será el asignado a este nuevo recurso. El único caso de error posible es que el usuario no tenga los permisos necesarios.
\item[•]Response:
{\codesize
\begin{verbatim}
{`error': `no', `id': `4'}
{`error': `si', `id': `-1'}
\end{verbatim}
}
\end{itemize}

\subsection{localhost/duocode/rest/temas/idTema}
En este apartado se explica cómo hacer un GET para obtener un tema en específico, un PUT para modificarlo y un DELETE para eliminarlo.
En estos requisitos obtenemos el ID del tema mediante la URL.
\vspace{1em}

\textbf{REQ03 – GET:} Devuelve los datos y las lecciones que tiene el tema.

\begin{itemize}
\item[•]Response:
{\codesize
\begin{verbatim}
{`titulo': `Bucles', `descripcion': `En este tema veremos bucles while', 
`fechaCreacion': `17/11/2014', `lecciones': 
[`localhost/duocode/rest/lecciones/1', `localhost/duocode/rest/lecciones/2'],
 `orden': `2'}
\end{verbatim}
}
\end{itemize}

\textbf{REQ04 – PUT:} Modifica el tema en su totalidad.
\begin{itemize}
\item[•]Payload:
{\codesize
\begin{verbatim}
{`titulo': `tituloTema', `descripcion': `descripcionTema', 
`orden': `int con el orden en el que lo quieres mostrar'}
\end{verbatim}
}

Enviamos por Header el ID del usuario, el token y el network, que nos sirven para comprobar que el usuario es administrador y tiene permisos para hacer esta operación:

\item[•]Header: 
{\codesize
\begin{verbatim}
{`idUsuario': `2', `token': `token', `network': `network'}
\end{verbatim}
}

La respuesta que obtenemos está conformada por ``error''. En caso afirmativo significa que algo ha fallado y no ha podido terminar la operación correctamente; en caso negativo todo ha ido correctamente. El único caso de error posible es que el usuario no tenga los permisos necesarios.

\item[•]Response: 
{\codesize
\begin{verbatim}
{`error': `no'}
\end{verbatim}
}
\end{itemize}

\textbf{REQ05 – DELETE:} Borra el tema completamente. Enviamos por Header el ID del usuario, el token y el network, que nos sirven para comprobar que el usuario es administrador y tiene permisos para hacer esta operación:
\begin{itemize}
\item[•]Header:
{\codesize
\begin{verbatim}
{`idUsuario': `2', `token': `token', `network': `network'}
\end{verbatim}
}

La respuesta que obtenemos está conformada por ``error'', en caso afirmativo significa que algo ha fallado y no ha podido terminar la operación correctamente; en caso negativo todo ha ido correctamente. El único caso de error posible es que el usuario no tenga los permisos necesarios.
\item[•]Response: 
{\codesize
\begin{verbatim}
{`error': `no'}
\end{verbatim}
}
\end{itemize}

\subsection{localhost/duocode/rest/lecciones/}
En este apartado se explica cómo hacer un GET para obtener todas las lecciones y un POST para crear una nueva lección.
\vspace{1em}

\textbf{REQ06 – GET:} Devuelve la lista de todas las lecciones existentes en modo URL. No hace falta enviarle ninguna información. 
\begin{itemize}
\item[•]Response:
{\codesize
\begin{verbatim}
{`lecciones': [`localhost/duocode/rest/lecciones/1', 
`localhost/duocode/rest/lecciones/2']}
\end{verbatim}
}
\end{itemize}

\textbf{REQ07 – POST:} Crea una nueva lección. Por una parte le pasamos por Payload la información de la lección que queremos crear (título, descripción, explicación que aparecerá al inicio de ésta, orden en el que queremos que se muestre, ID del tema al que pertenecerá, un array de ejercicios que compondrán la lección y otro array de lecciones de las que depende para ser desbloqueada):

\begin{itemize}
\item[•]Payload: 
{\codesize
\begin{verbatim}
{`titulo': `titulo lección', `descripcion': `descripción lección', 
`explicacion': `explicación detallada', `orden': `4', `idTema': `1', 
`idEjercicios': [`8', `14'], `leccionesDesbloqueadoras': [`1', `2']}
\end{verbatim}
}

Por otra parte le enviamos por Header el ID del usuario, el token y el network, que nos sirven para comprobar que el usuario es administrador y tiene permisos para hacer esta operación:
\item[•]Header: 
{\codesize
\begin{verbatim}
{`idUsuario': `2', `token': `token', `network': `network'}
\end{verbatim}
}

La respuesta que obtenemos está conformada por ``error'' e ``id''. Si la respuesta en el campo ``error'' es afirmativa significa que algo ha fallado y no ha podido terminar la operación correctamente, en este caso el id será ``-1'' ya que no ha sido asignado ninguno; si es negativa todo ha ido correctamente y el id será el asignado a esta nueva lección. El único caso de error posible es que el usuario no tenga los permisos necesarios.
\item[•]Response: 
{\codesize
\begin{verbatim}
{`error': `no', `id': `3'}
{`error': `si', `id': `-1'}
\end{verbatim}
}
\end{itemize}

\subsection{localhost/duocode/rest/lecciones/idLeccion}
En este apartado se explica cómo hacer un GET para obtener una lección en específica, un PUT para modificarla y un DELETE para eliminarla.
En estos requisitos obtenemos el ID del candidato mediante la URL.
\vspace{1em}


\textbf{REQ08 – GET:} Devuelve los datos de la lección.

\begin{itemize}
\item[•]Response:
{\codesize
\begin{verbatim}
{`titulo': `título de la lección (ej. Bucles fácil)', 
`descripcion': `descripción de la lección (ej. bucles para practicar)', 
`explicacion': `explicación detallada', `fechaCreacion': `17/11/2014',
`ejercicios': [`localhost/duocode/rest/ejercicios/2', 
`localhost/duocode/ejercicios/3'], 
`leccionesDesbloqueadoras': [`1', `2'], `orden': `3'}
\end{verbatim}
}
\end{itemize}

\textbf{REQ09 – PUT:} Modifica una lección. Aparte de cambiar los datos de una lección, en este método también podemos marcar como completada la lección para un usuario en un determinado lenguaje, por lo que tenemos dos posibles Payloads:

\begin{itemize}
\item[•]Payload para modificar lección:
{\codesize
\begin{verbatim}
{'leccion' : {`titulo': `titulo lección', `descripcion': 
`descripción lección', `explicacion': `explicación detallada', 
`idEjercicios': [`1', `2'], `leccionesDesbloqueadoras': [`2', `3'], 
`orden': `2', `idTema': `1'}}
\end{verbatim}
}

\item[•]
Payload para marcar como lección completada para un usuario:
{\codesize
\begin{verbatim}
{`idUsuarioCompletaLeccion': `3', `lenguajeCompletadoLeccion': `Java', 
`leccion': {`titulo': `titulo lección', 
`descripcion': `descripción de la lección', 
`explicacion':`explicación detallada', `idEjercicios': [`1', `2'], 
`leccionesDesbloqueadoras': [`2', `3'], `orden': `2', `idTema': `1'}}
\end{verbatim}
}

Enviamos por Header el ID del usuario, el token y el network, que nos sirven para comprobar que el usuario es administrador y tiene permisos para hacer esta operación:

\item[•]Header:
{\codesize
\begin{verbatim}
{`idUsuario': `2', `token': `token', `network': `network'}
\end{verbatim}
}

La respuesta que obtenemos está conformada por ``error''. En caso afirmativo significa que algo ha fallado y no ha podido terminar la operación correctamente; en caso negativo todo ha ido correctamente. El único caso de error posible es que el usuario no tenga los permisos necesarios.
\item[•]Response:
{\codesize
\begin{verbatim} 
{`error': `no'}
\end{verbatim}
}
\end{itemize}

\textbf{REQ10 – DELETE:} Borra la lección completamente. Enviamos por Header el ID del usuario, el token y el network, que nos sirven para comprobar que el usuario es administrador y tiene permisos para hacer esta operación:
\begin{itemize}

\item[•]
Header: 
{\codesize
\begin{verbatim}
{`idUsuario': `2', `token': `token', `network': `network'}
\end{verbatim}
}

La respuesta que obtenemos está conformada por ``error''. En caso afirmativo significa que algo ha fallado y no ha podido terminar la operación correctamente; en caso negativo todo ha ido correctamente. El único caso de error posible es que el usuario no tenga los permisos necesarios.

\item[•] 
Response:
{\codesize
\begin{verbatim} 
{`error': `no'}
\end{verbatim}
}
\end{itemize}

\subsection{localhost/duocode/rest/ejercicios}
En este apartado se explica cómo hacer un GET para obtener todos los ejercicios y un POST para crear un nuevo ejercicio.
\vspace{1em}

\textbf{REQ11 – GET:} Devuelve la lista de todos los ejercicios existentes en modo URL. No hace falta enviarle ninguna información. 

\begin{itemize}
\item[•]
Response: 
{\codesize
\begin{verbatim}
{`ejercicios': [`localhost/duocode/rest/temas/ejercicios/1', 
`localhost/duocode/rest/ejercicios/2']}
\end{verbatim}
}
\end{itemize}

\textbf{REQ12 – POST:} Crea un ejercicio nuevo. Por una parte le pasamos el nombre y los enunciados (un mismo ejercicio puede tener varios enunciados porque cada uno corresponde a distintos lenguajes de programación):

\begin{itemize}
\item[•]
Payload: 
{\codesize
\begin{verbatim}
{`nombre': `nombreDelEjercicio', `enunciados': [`1', `2']}
\end{verbatim}
}

Por otra parte le enviamos por Header el ID del usuario, el token y el network, que nos sirven para comprobar que el usuario es administrador y tiene permisos para hacer esta operación:

\item[•]
Header: 
{\codesize
\begin{verbatim}
{`idUsuario': `2', `token': `token', `network': `network'}
\end{verbatim}
}

La respuesta que obtenemos está conformada por ``error'' e ``id''. Si la respuesta en el campo ``error'' es afirmativa significa que algo ha fallado y no ha podido terminar la operación correctamente, en este caso el id será ``-1'' ya que no ha sido asignado ninguno; si es negativa todo ha ido correctamente y el id será el asignado a este nuevo recurso. El único caso de error posible es que el usuario no tenga los permisos necesarios.

\item[•]
Response: 
{\codesize
\begin{verbatim}
{`error': `no', `id': `4'}
{`error': `si', `id': `-1'}
\end{verbatim}
}
\end{itemize}

\subsection{localhost/duocode/rest/ejercicios/idEjercicio}
En este apartado se explica cómo hacer un GET para obtener un ejercicio en específico, un PUT para modificarlo y un DELETE para eliminarlo.
En estos requisitos obtenemos el ID del ejercicio mediante la URL.
\vspace{1em}

\textbf{REQ13 – GET:} Devuelve los datos del ejercicio, los enunciados que tiene y el nombre del lenguaje asociado a cada uno.

\begin{itemize}
\item[•]
Response: 
{\codesize
\begin{verbatim}
{`nombre': `nombreDelEjercicio', `fechaCreacion': `17/11/2014', 
`enunciados': [{`enunciado': `localhost/duocode/rest/enunciados/5', 
`nombreLenguaje': `Java'}, {`enunciado': 
`localhost/duocode/rest/enunciados/8', `nombreLenguaje': `C++'}]}
\end{verbatim}
}
\end{itemize}

\textbf{REQ14 – DELETE:} Borra un ejercicio completamente. Enviamos por Header el ID del usuario, el token y el network, que nos sirven para comprobar que el usuario es administrador y tiene permisos para hacer esta operación:

\begin{itemize}
\item[•]
Header: 
{\codesize
\begin{verbatim}
{`idUsuario': `2', `token': `token', `network': `network'}
\end{verbatim}
}

La respuesta que obtenemos está conformada por ``error''. En caso afirmativo significa que algo ha fallado y no ha podido terminar la operación correctamente; en caso negativo todo ha ido correctamente. El único caso de error posible es que el usuario no tenga los permisos necesarios.
\item[•]
Response: 
{\codesize
\begin{verbatim}
{`error': `no'}
\end{verbatim}
}
\end{itemize}

\subsection{localhost/duocode/rest/enunciados}
En este apartado se explica cómo hacer un GET para obtener todos los enunciados y un POST para crear un nuevo enunciado.
\vspace{1em}

\textbf{REQ15 – GET:} Devuelve la lista de todos los enunciados existentes en modo URL. No hace falta enviarle ninguna información.

\begin{itemize}
\item[•]
Response:
{\codesize
\begin{verbatim}
{`enunciados': [{`enunciado': `localhost/duocode/rest/enunciados/1', 
`nombreLenguaje': `Java'}, {`enunciado': `localhost/duocode/rest/enunciados/2', 
`nombreLenguaje': `C++'}]}
\end{verbatim}
}

\end{itemize}

\textbf{REQ16 – POST:} Crea un enunciado nuevo. Por una parte le pasamos el lenguaje, el código y el ID del ejercicio correspondiente:

\begin{itemize}
\item[•]
Payload: 
{\codesize
\begin{verbatim}
{`nombreLenguaje': `Java', `codigo': `código del enunciado', 
`idDelEjercicioQueResuelve': `1'}
\end{verbatim}
}

Por otra parte le enviamos por Header el ID del usuario, el token y el network, que nos sirven para comprobar que el usuario es administrador y tiene permisos para hacer esta operación:

\item[•]
Header:
{\codesize
\begin{verbatim} 
{`idUsuario': `2', `token': `token', `network': `network'}
\end{verbatim}
}

La respuesta que obtenemos está conformada por ``error'' e ``id''. Si la respuesta en el campo ``error'' es afirmativa significa que algo ha fallado y no ha podido terminar la operación correctamente, en este caso el id será ``-1'' ya que no ha sido asignado ninguno; si es negativa todo ha ido correctamente y el id será el asignado a este nuevo enunciado. El único caso de error posible es que el usuario no tenga los permisos necesarios.

\item[•]
Response: 
{\codesize
\begin{verbatim}
{`error': `no', `id': `4'} 
{`error': `si', `id': `-1'} 
\end{verbatim}
}
\end{itemize}

\subsection{localhost/duocode/rest/enunciados/idEnunciado}
En este apartado se explica cómo hacer un GET para obtener un enunciado en específico, un PUT para modificar un enunciado y un DELETE para eliminarlo.
En estos requisitos obtenemos el ID del candidato mediante la URL.
\vspace{1em}


\textbf{REQ17 – GET:} Devuelve los datos del enunciado.

\begin{itemize}
\item[•]
Response: 
{\codesize
\begin{verbatim}
{`fechaCreacion': `17/11/2014', `codigo': `código del enunciado a resolver', 
`nombreLenguaje': `Java', `idDelEjercicioQueResuelve': `1'}
\end{verbatim}
}
\end{itemize}

\textbf{REQ18 – PUT:} Modifica un enunciado.

\begin{itemize}
\item[•]
Payload: 
{\codesize
\begin{verbatim}
{`nombreLenguaje': `Java', `codigo': `código del enunciado', 
`idDelEjercicioQueResuelve': `1'}
\end{verbatim}
}

Enviamos por Header el ID del usuario, el token y el network, que nos sirven para comprobar que el usuario es administrador y tiene permisos para hacer esta operación:
\item[•]
Header: 
{\codesize
\begin{verbatim}
{`idUsuario': `2', `token': `token', `network': `network'}
\end{verbatim}
}

La respuesta que obtenemos está conformada por ``error''. En caso afirmativo significa que algo ha fallado y no ha podido terminar la operación correctamente; en caso negativo todo ha ido correctamente. El único caso de error posible es que el usuario no tenga los permisos necesarios.
\item[•]
Response: 
{\codesize
\begin{verbatim}
{`error': `no'}
\end{verbatim}
}
\end{itemize}

\textbf{REQ19 – DELETE:} Borra un enunciado completamente. Enviamos por Header el ID del usuario, el token y el network, que nos sirven para comprobar que el usuario es administrador y tiene permisos para hacer esta operación:
\begin{itemize}
\item[•]
Header:
{\codesize
\begin{verbatim} 
{`idUsuario': `2', `token': `token', `network': `network'}
\end{verbatim}
}

La respuesta que obtenemos está conformada por ``error''. En caso afirmativo significa que algo ha fallado y no ha podido terminar la operación correctamente; en caso negativo todo ha ido correctamente. El único caso de error posible es que el usuario no tenga los permisos necesarios.
\item[•]
Response: 
{\codesize
\begin{verbatim}
{`error': `no'}
\end{verbatim}
}
\end{itemize}

\subsection{localhost/duocode/rest/lenguajes/}
En este apartado se explica cómo hacer un GET para obtener todos los lenguajes y un POST para crear un nuevo lenguaje.
\vspace{1em}

\textbf{REQ20 – GET:} Devuelve la lista de todos los lenguajes existentes. No hace falta enviarle ninguna información.
  
\begin{itemize}
\item[•]
Response:
{\codesize
\begin{verbatim} 
{`lenguajes': [{`nombre': `Java'}, {`nombre': `C++'}]}
\end{verbatim}
}
\end{itemize}

\textbf{REQ21 – POST:} Crea un lenguaje nuevo. La única información necesaria es el nombre.

\begin{itemize}
\item[•]
Payload: 
{\codesize
\begin{verbatim}
{`nombre': `Python'}
\end{verbatim}
}

Enviamos por Header el ID del usuario, el token y el network, que nos sirven para comprobar que el usuario es administrador y tiene permisos para hacer esta operación:

\item[•]
Header: 
{\codesize
\begin{verbatim}
{`idUsuario': `2', `token': `token', `network': `network'}
\end{verbatim}
}

La respuesta que obtenemos está conformada por ``error'' y ``nombreConfirmacion''. Si la respuesta en el campo ``error'' es afirmativa significa que algo ha fallado y no ha podido terminar la operación correctamente; si es negativa todo ha ido correctamente y el ``nombreConfirmacion'' será el asignado a este nuevo lenguaje. El único caso de error posible es que el usuario no tenga los permisos necesarios.

\item[•]
Response: 
{\codesize
\begin{verbatim}
{`error': `no', `nombreConfirmacion': `Python'}
\end{verbatim}
}
\end{itemize}

\subsection{localhost/duocode/rest/candidatos/}
En este apartado se explica cómo hacer un GET para obtener todos los candidatos y un POST para crear un nuevo candidato.
\vspace{1em}

\textbf{REQ22 – GET:} Devuelve la lista de todos los candidatos existentes en modo URL. No hace falta enviarle ninguna información. 

\begin{itemize}
\item[•]
Response: 
{\codesize
\begin{verbatim}
{`candidatos': [`localhost/duocode/rest/candidatos/1', 
`localhost/duocode/rest/candidatos/2']}
\end{verbatim}
}
\end{itemize}

\textbf{REQ23 – POST:} Crea un nuevo candidato asociado al usuario que solicita la petición. Por un lado le enviamos el código del candidato, el ID del ejercicio que resuelve, el lenguaje en el que está escrito el candidato y el lenguaje del enunciado:
\begin{itemize}
\item[•]
Payload: 
{\codesize
\begin{verbatim}
{`codigo': `codigoDelCandidato', `idEjercicio': `4', 
`nombreLenguajeDestino': `Java', `nombreLenguajeOrigen': `C++'}
\end{verbatim}
}

Por otra parte le enviamos por Header el ID del usuario, el token y el network, que nos sirven para saber qué usuario es el que ha enviado el candidato:
\item[•]
Header: 
{\codesize
\begin{verbatim}
{`idUsuario': `2', `token': `token', `network': `network'}
\end{verbatim}
}
La respuesta que obtenemos está conformada por ``error'' e ``idCandidato''. Si la respuesta en el campo ``error'' es afirmativa significa que algo ha fallado y no ha podido terminar la operación correctamente, en este caso el id será ``-1'' ya que no ha sido asignado ninguno; si es negativa todo ha ido correctamente y el id será el asignado a este nuevo candidato. 
\item[•]
Response: 
{\codesize
\begin{verbatim}
{`error': `no', `id': `4'}
\end{verbatim}
}
\end{itemize}

\subsection{localhost/duocode/rest/candidatos/idCandidato}
En este apartado se explica cómo hacer un GET para obtener un candidato en específico, un PUT para modificarlo y un DELETE para eliminarlo.
En estos requisitos obtenemos el ID del candidato mediante la URL.
\vspace{1em}

\textbf{REQ24 – GET:} Devuelve los datos de un candidato, incluidos los votos que tiene:

\begin{itemize}
\item[•]
Response: 
{\codesize
\begin{verbatim}
{'idEjercicio': `localhost/duocode/rest/ejercicios/4', 
`nombreLenguajeOrigen': `Java', `nombreLenguajeDestino: `C++', 
`codigo': `código del candidato', `idUsuarioCreador': `2', 
`fechaCreacion': `17/11/2014', `votos': [{`idUsuarioVoto': `8', 
`voto': `pos'}, {`idUsuarioVoto':`5', `voto': `neg'}]}
\end{verbatim}
}
\end{itemize}

\textbf{REQ25 – PUT:} Excepcionalmente no modifica todo el candidato, sino que sirve para que un usuario pueda votar, modificar el voto o eliminarlo (si se vuelve a votar positivo o negativo el voto se anula). Se envía el ID del usuario y el voto (1 si es positivo y 0 si es negativo).

\begin{itemize}
\item[•]
Payload: 
{\codesize
\begin{verbatim}
{`votar': {`idUsuario': `6', `voto': `1'}}
\end{verbatim}
}

La respuesta que obtenemos está conformada por ``error''. En caso afirmativo significa que algo ha fallado y no ha podido terminar la operación correctamente; en caso negativo todo ha ido correctamente. 
\item[•]
Response:
{\codesize
\begin{verbatim} 
{`error': `no'}
\end{verbatim}
}
\end{itemize}

\textbf{REQ26 – DELETE:} Borra un candidato completamente.

\begin{itemize}
\item[•]
Header:
{\codesize
\begin{verbatim} 
{`idUsuario': `2', `token': `token', `network': `network'}
\end{verbatim}
}

La respuesta que obtenemos está conformada por ``error''. En caso afirmativo significa que algo ha fallado y no ha podido terminar la operación correctamente; en caso negativo todo ha ido correctamente. El único caso de error posible es que el usuario no tenga los permisos necesarios.

\item[•]
Response: 
{\codesize
\begin{verbatim}
{`error': `no'}
\end{verbatim}
}
\end{itemize}

\subsection{localhost/duocode/rest/usuarios/}
En este apartado se explica cómo hacer un GET para obtener todos los usuarios.
\vspace{1em}

\textbf{REQ27 – GET:} Devuelve todos los usuarios. A diferencia de otros, este GET solo lo puede hacer un administrador por lo que necesitamos nuevamente del Header.

\begin{itemize}
\item[•]
Header:
{\codesize
\begin{verbatim}
{`idUsuario': `2', `token': `token', `network': `network'}
\end{verbatim}
}

La respuesta que obtenemos está conformada por ``error'' y la lista de usuarios. Si ``error'' tiene una respuesta afirmativa, significa que algo ha fallado y no ha podido terminar la operación correctamente; en caso negativo todo ha ido correctamente y muestra la lista de usuarios mediante su URL. El único caso de error posible es que el usuario no tenga los permisos necesarios.

\item[•]
Response: 
{\codesize
\begin{verbatim}
{`error': `no', `usuarios': [`localhost/duocode/rest/usuario/1', 
`localhost/duocode/rest/usuario/2']}
\end{verbatim}
}
\end{itemize}

\subsection{localhost/duocode/rest/usuarios/idUsuario}
En este apartado se explica cómo hacer un GET para obtener un usuario en específico y un DELETE para eliminarlo.
En estos requisitos obtenemos el ID del candidato mediante la URL.
\vspace{1em}

\textbf{REQ28 – GET:} Devuelve la información asociada a un usuario. Enviamos por Header el ID del usuario, el token y el network, que nos sirven para comprobar que el usuario que accede es el mismo del que se da la información:

\begin{itemize}
\item[•]
Header: 
{\codesize
\begin{verbatim}
{`idUsuario': `2', `token': `token', `network': `network'}
\end{verbatim}
}

La respuesta es la información detallada de toda la sesión del usuario
\item[•]
Response: 
{\codesize
\begin{verbatim}
{`nick': `nickDelUsuairo', `leccionesCompletadas': 
[`localhost/duocode/rest/lecciones/1', `localhost/duocode/rest/lecciones/2'], 
`favoritos': [{`ejercicio': `localhost/duocode/rest/ejercicios/5', 
`nombreLenguajeOrigen': `Java', `nombreLenguajeDestino': `C++'}, 
{`ejercicio': `localhost/duocode/rest/ejercicios/7', 
`nombreLenguajeOrigen': `Python', `nombreLenguajeDestino': `C++'}], 
`historialEjercicios': [{`idEnvio': `1', 
`ejercicio': `localhost/duocode/rest/ejercicios/4', 
`nombreLenguajeOrigen': `Java', `nombreLenguajeDestino': `C++', 
`codigo': `código enviado', `fecha': `17/11/2014', `puntuacion': `2'}, 
{`idEnvio': `2', `ejercicio': `localhost/duocode/rest/ejercicios/9', 
`nombreLenguajeOrigen': `Python', `nombreLenguajeDestino': `Perl', 
`codigo': `código enviado', `fecha': `18/11/2014', `puntuacion': `7'}], 
`candidatosPropuestos': [`localhost/duocode/rest/candidatos/18', 
`localhost/duocode/rest/candidatos/23']}
\end{verbatim}
}
\end{itemize}

\textbf{REQ29 – DELETE}: Borra un usuario completamente.
Enviamos por Header el ID del usuario, el token y el network, que nos sirven para comprobar que el usuario es administrador y tiene permisos para hacer esta operación:

\begin{itemize}
\item[•]
Header: 
{\codesize
\begin{verbatim}
{`idUsuario': `2', `token': `token', `network': `network'}
\end{verbatim}
}

La respuesta que obtenemos está conformada por ``error''. En caso afirmativo significa que algo ha fallado y no ha podido terminar la operación correctamente; en caso negativo todo ha ido correctamente. El único caso de error posible es que el usuario no tenga los permisos necesarios.

\item[•]
Response: 
{\codesize
\begin{verbatim}
{`error': `no'}
\end{verbatim}
}
\end{itemize}

\subsection{localhost/duocode/rest/envios}
En este apartado se explica cómo hacer un GET para obtener todos los envíos y un POST para crear un nuevo envío.
\vspace{1em}

\textbf{REQ30 – GET}: Devuelve todos los envíos para que un administrador pueda tener información de ellos. Como solo puede tener acceso a esto el administrador, es necesaria la siguiente información:

\begin{itemize}
\item[•]
Header: 
{\codesize
\begin{verbatim}
{`idUsuario': `2', `token': `token', `network': `network'}
\end{verbatim}
}

La respuesta se compone de usuarios con historiales de ejercicios.

\item[•]
Response: 
{\codesize
\begin{verbatim}
{`envios': [{`idUsuario': `1', `historialEjercicios': [{`idEnvio': `1', 
`ejercicio': `localhost/duocode/rest/ejercicios/5', 
`nombreLenguajeOrigen': `Java', `nombreLenguajeDestino': `C++', 
`codigo': `código enviado', `fecha': `17/11/2014', `puntuacion': `2'}, 
{`idEnvio': `2', `ejercicio': `localhost/duocode/rest/ejercicios/7', 
`nombreLenguajeOrigen': `Python', `nombreLenguajeDestino': `Perl', 
`codigo': `código enviado', `fecha': `18/11/2014', `puntuacion': `7'}]}, 
{`idUsuario': `4', `historialEjercicios': [{`idEnvio': `1', 
`ejercicio': `localhost/duocode/rest/ejercicios/6', 
`nombreLenguajeOrigen': `Java', `nombreLenguajeDestino': `C++', 
`codigo': `código enviado', `fecha': `17/11/2014', `puntuacion': `2'}, 
{`idEnvio': `2', `ejercicio': `localhost/duocode/rest/ejercicios/5', 
`nombreLenguajeOrigen': `Python', `nombreLenguajeDestino': `Perl', 
`codigo': `código enviado', `fecha': `18/11/2014', `puntuacion': `7'}]}]}
\end{verbatim}
}
\end{itemize}

\textbf{REQ31 – PUT:} Corrige un ejercicio y también se comprueba si se ha completado la lección; en caso afirmativo se marca como completada en la base de datos. Por una parte enviamos la URL del ejercicio, el lenguaje del enunciado, el lenguaje de la solución y el código:

\begin{itemize}
\item[•]
Payload: 
{\codesize
\begin{verbatim}
{`ejercicio': `localhost/duocode/rest/ejercicios/idEjercicio1', 
`nombreLenguajeOrigen': `Java', `nombreLenguajeDestino': `C++', 
`codigo': `código enviado en el lenguaje de destino'} 
\end{verbatim}
}

Por otra parte enviamos el idUsuaro, el token y el network para comprobar que el usuario que envía el ejercicio para corregir es el que tiene la sesión iniciada.
\item[•]
Header: 
{\codesize
\begin{verbatim}
{`idUsuario': `2', `token': `token', `network': `network'}
\end{verbatim}
}

La respuesta que obtenemos está conformada por ``error'' y ``puntuacion''. Si ``error'' tiene un valor afirmativo significa que algo ha fallado y no ha podido terminar la operación correctamente; en caso negativo todo ha ido correctamente y devuelve la puntuación que ha obtenido el ejercicio al ser corregido. El único caso de error posible es que el usuario no coincida.
\item[•]
Response: 
{\codesize
\begin{verbatim}
{`error': `no', `puntuacion': `2'}
\end{verbatim}
}
\end{itemize}
