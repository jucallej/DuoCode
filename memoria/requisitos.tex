%!TEX root = memoria_duocode_interfaz.tex
\subsection{Requisitos}
Esta es la primera fase que abarcamos al empezar con el proyecto. Después de discutir las ideas sobre cómo queríamos que fuese el servicio web, empezamos a redactar los requisitos. Hemos ido modificándolos a medida que nos encontrábamos con algo distinto a lo que nos habíamos imaginado o que no habíamos tenido en cuenta antes de empezar a implementar.
\vspace{1em}

La redacción de los requisitos nos ha facilitado mucho la tarea de la implementación ya que así no teníamos que improvisar mientras escribíamos código y cuando lo necesitábamos recurríamos a ellos.
\vspace{1em}

Para hacerlos, los organizamos según los recursos que utilizaríamos (por ejemplo: temas, lecciones, ejercicios, usuarios...) y cada uno de ellos tiene un máximo cuatro de partes (GET, POST, PUT y DELETE) debido a que utilizamos una API REST y trabajamos con estos métodos de petición. Indicamos en todos qué se tiene que recibir por Payload y por Header (opcionalmente) y cuál será la respuesta que recibiremos.

