%!TEX root = memoria_duocode_interfaz.tex
The project ``Desarrollo de un front-end para DuoCode'' aims to develop a web application for learning programming languages. This application allows users to learn programming languages from the ones they already know by overcoming different lessons and subjects. At the beginning, only a few lessons of each subject are available; the other ones will be unlocked as the above are completed.

To get a clear structure, subjects in DuoCode are a series of lessons. Also, lessons consist of a collection of exercises, which are based on a statement in the language that the user already knows and that will have to be translated into the language he wants to learn.

As the user solves the exercises, his score increases. In addition, the user has lives, which are subtracted when the answer given is not correct. This gives the application an enjoyable type of learning, making it look like a game. DuoCode allows the user to mark exercises as favorite to keep them accessible and to consult them at any time. 

Furthermore, as a snippet in a specific language may be written in different ways, a part of the application is dedicated to \textbf{candidates}. If a user fails to resolve an exercise but thinks his solution is right, he has the option of sending his exercise as a candidate. Thus, the solution becomes evaluated by other users. If it gets enough positive votes and an administrator considers it valid, it becomes a right solution for this exercise; on the other hand, if it gets negative votes this solution will be discarded and cannot be proposed again.

In order to log in, the user does not have to register on the website because it includes a login with \textbf{Facebook} and \textbf{Google+}. Therefore, the only thing needed to use the application is to grant access permissions to basic user profile information of the corresponding network. Moreover, DuoCode has the ability to share on Facebook the achievements after overcoming a lesson.

Finally, DuoCode is a useful tool not only for students but also for teachers, as its database also stores information about the submissions made by the students.

\textbf{Keywords:} Web, REST, AngularJS, database, Bootstrap, learning, programming languages, translation.