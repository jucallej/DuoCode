%!TEX root = memoria_duocode_interfaz.tex
The project ``Front-end development for Duocode'' aims to develop a web application for learning programming languages. This application allows users to learn programming languages from the ones they already know by overcoming different lessons and subjects. At the beginning, only a few lessons of each subject are available; the other ones will be unlocked as the above are completed.

To get a clear structure, subjects in DuoCode are a series of lessons. Also, lessons consist of a collection of exercises, which are based on a statement in the language that the user knows and that will have to be translated into the language he wants to learn.

As the user is solving the exercises, his score increases. In addition, the user has lives, which are subtracted when the answer given is not correct. This gives the application a enjoyable type of learning, making it look like a game. Duocode allows the user to mark exercises as favorite to keep them accessible and to consult them at any time. 

Furthermore, as a snippet in a specific language may be written in different ways, a part of the application is dedicated to \textbf{candidates}. If a user fails to resolve an exercise but think his solution is right, he has the option of sending his exercise as a candidate. In doing so, the solution becomes evaluated by other users. If it gets enough positive votes and an administrator considered it valid, it becomes right solution of this exercise from that time; on the other hand, if it gets negative votes this solution will be discarded and cannot be proposed again.

In order to log in, the user doesn't have to register on the website because it includes a login with \textbf{Facebook} and \textbf{Google+}. Therefore, the only thing needed to use the application is to give access permissions to basic user profile information of the corresponding network. Also, DuoCode has the ability to share on Facebook the success after overcoming a lesson.

Finally, DuoCode is a useful tool not only for students but also for teachers, as its database also stores information on sendings made by the students.