%!TEX root = memoria_duocode_interfaz.tex

En este proyecto hemos desarrollado \textit{DuoCode}, un sistema de aprendizaje colaborativo de lenguajes de programación mediante traducciones, tomando como referencia un lenguaje conocido o incluso el lenguaje natural, en el caso de que el usuario no tenga conocimientos previos de programación. 

Durante nuestro paso por la universidad tuvimos que realizar diversos proyectos para asignaturas que nos aportaron conocimientos sobre temas específicos. Con el desarrollo de \textit{DuoCode} pudimos poner en práctica todos los conceptos aprendidos durante nuestra formación en un único proyecto.
%
Además, nos sirvió para aprender tecnologías que no se enseñan en la universidad y cuentan con mucha demandada actualmente como \textit{AngularJS}, \textit{BootStrap}, \textit{Jersey} o \textit{PhoneGap}.

Optamos por un proceso de desarrollo ágil, dividido en sprints de una media de dos semanas de duración que finalizaban con una reunión de revisión con los tutores. Ocasionalmente nos reunimos los alumnos para tratar problemas específicos de diseño o implementación. Esta metodología de desarrollo también fue nueva para nosotros, ya que en la mayoría de las prácticas y proyectos que realizamos en la universidad seguían sistemas tradicionales como el modelo de cascada o el modelo en espiral. El resultado es satisfactorio y nos sirvió para ver cómo se desarrolla software en la actualidad, pues un gran porcentaje de las empresas de desarrollo de aplicaciones móviles y webs usan metodologías ágiles.

Además aprendimos a desarrollar un proyecto en equipo usando \textit{Git}, un sistema de control de versiones (\textit{VCS}) muy demandado y exigido por cualquier empresa de desarrollo. En este caso la curva de aprendizaje sí que fue elevada, algúnos de los componentes no habían usado nunca un \textit{VCS} y al principio resultó algo complicado, pero mereció la pena el tiempo que invertimos en su aprendizaje y a día de hoy volveríamos a elegirlo sin duda.

El tema que más tiempo nos quitó fue la configuración del servidor \textit{GlassFish} para que usase el protocolo \texttt{HTTPS} y poder cifrar el intercambio de información. Tuvimos varios problemas para configurarlo correctamente por nuestra falta de conocimientos sobre el tema, además no encontramos mucha documentación actual por internet. Por todo eso decidimos cambiar el servidor en el que alojar el servicio web por un servidor TomCat, ya que cuenta con las mismas características que \textit{GlassFish} y hay mucha más documentación online. Esta decisión supuso un acierto y nos permitió seguir desarrollando usando el protocolo \texttt{HTTPS}.

El balance final del desarrollo de \textit{DuoCode} es positivo. Nos ha aportado conocimientos durante el desarrollo y es una herramienta de aprendizaje a través de la cual los futuros estudiantes de informática podrán aprender un lenguaje nuevo o incluso aprender un lenguaje desde cero.
Varios colegios e institutos españoles están empezando a implantar la programación como una asignatura obligatoria, y herramientas como \textit{DuoCode} ayudarán a los alumnos a practicar los conceptos aprendidos durante las clases.

El trabajo presentado en este proyecto nos ofrece una buena base para futuras extensiones que nos permitan mejorar su usabilidad y generalidad. El uso de patrones de diseño como el \textit{Abstract Mapper} o el framework \textit{AngularJS} permitieron desarrollar un código mantenible y extensible para futuros desarrolladores. Aunque la versión actual de \textit{DuoCode} es estable y cumple toda la funcionalidad descrita al comienzo de la memoria, también hay mucho trabajo futuro que desarrollar:
\begin{itemize}
\item Integración de \textbf{insignias digitales}. Una `insignia digital' es el registro en línea de un logro, cualidad, calidad o interés. Se puede encontrar más información en el siguiente \textit{enlace}: \url{https://support.mozilla.org/es/kb/que-es-una-insignia}

\item Integración de DuoCode en el ambiente universitario y valorar las insignias conseguidas.

\item Mostrar qué parte de código está mal en cada corrección.

\item Intercalar preguntas en distintos lenguajes.

\item Agrupar lenguajes según el paradigma de programación: Lenguajes imperativos, funcionales, lógicos, declarativos...

\item Coloreado de la sintaxis en tiempo real.

\item Posibilidad de cambiar la paleta de colores en el diseño.

\item Refinamiento de la versión móvil.

\item Normalización de la base de datos.

\end{itemize}