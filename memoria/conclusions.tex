%!TEX root = memoria_duocode_interfaz.tex

In this project we have developed \textit{DuoCode}, a new way of learning to code. It consists of a platform where users can learn new programming languages by translating small pieces of code from a known one. It can even be the natural language if the user have never studied any programming language before.

During our years in the university we have acquired knowledge about different areas related to computer science. Developing \textit{DuoCode} allowed us to apply all our knowledge into a single project. However we also had to learn new useful technologies that helped us growing as an engineer, such as \textit{AngularJS}, \textit{BootStrap}, \textit{Jersey} or \textit{PhoneGap}.

First of all we decided to use an agile development methodology. Every sprint lasted 2 weeks and ended with a meeting where we talked about what we did. It was a new way of developing for us, we were used to use traditional methodologies such as \textit{Waterfall} or \textit{Spiral Development}. But it was a great decision, we have had very good results and we have learned to use one of the most popular ways of development.

We also have learned how to work in team by using \textit{Git}, the most common distributed version control. It took us some weeks to learn \textit{Git}, it isn't easy and it has a lot of features. However it is very useful once you know how to use the basic commands.

We had a lot of problems trying to configure SSL on \textit{GlassFish} in order to use \texttt{HTTPS} protocol. It is probably the thing that took most of our time so we decided to use \textit{TomCat} instead. \textit{TomCat} is similar to \textit{GlassFish} but it has a lot more documentation on the internet. Finally we could configure the server to use \texttt{HTTPS} protocol.

We are proud of the final result. We have learned new technologies and \textit{DuoCode} is a good way for future students to learn the basics of coding in a new language. Nowadays a lot of schools are interested on teaching how to code. \textit{DuoCode} is a very useful platform where students can practice and apply their knowledge.

\textit{DuoCode} can be extended on many ways. This project constitutes a good base for future extensions. By using design patterns such as \textit{Abstract Mapper} or \textit{AngularJS} framework we have obtained maintainable code, making it easier to future developers to extend the project. Although the actual version of \textit{DuoCode} is stable, a lot of new features can be implemented:

\begin{itemize}
\item \textit{Open Badges} integration: a new online standard to recognize and verify learning. More info can be found here: \url{http://openbadges.org/}

\item Show which part of the code is wrong.

\item Intercalate questions in different languages.

\item Group languages with same programming paradigm.

\item Highlight syntax in real time.

\item New designs.

\item Improve the mobile version.

\item Database normalization.

\end{itemize}