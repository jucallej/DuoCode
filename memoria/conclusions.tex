%!TEX root = memoria_duocode_interfaz.tex

In this project we have developed \textit{DuoCode}, a new way for learning how to code. It consists of a platform where users can learn new programming languages by translating small pieces of code from a known one. It can even be the natural language if the user has never studied any programming language before.

During our years in the university we have acquired knowledge about different areas related to computer science. Developing \textit{DuoCode} allowed us to apply all our knowledge into a single project. Moreover, we also learned new technologies that helped us growing as engineers, such as \textit{AngularJS}, \textit{BootStrap}, \textit{Jersey} or \textit{PhoneGap}.

First of all we decided to use an agile development methodology. Every sprint lasted 2 weeks and ended with a meeting where we talked about what we did. It was a new way of developing for us; we were used to apply traditional methodologies such as \textit{Waterfall} or \textit{Spiral Development}. It was a great decision, since we have had very good results and we have learned how to use one of the most popular ways of development nowadays.

We have also learned how to work in team by using \textit{Git}, the most common distributed version control. It took us some weeks to learn \textit{Git}, it is not simple and has a lot of features.
However, it was very useful once we learned how to use the basic commands.

We had a lot of problems trying to configure SSL on \textit{GlassFish} in order to use \texttt{HTTPS} protocol. It was probably the task that took most of our time so we decided to switch to \textit{TomCat}. \textit{TomCat} is similar to \textit{GlassFish} but it has much more documentation. Finally, we could configure the server to use the \texttt{HTTPS} protocol.

We are proud of the final result. We have learned new technologies and \textit{DuoCode} is a good way for future students to learn the basics of coding in a new language. Nowadays a lot of schools are interested in teaching how to code. \textit{DuoCode} is a very useful platform where students can practice and apply their knowledge.

\textit{DuoCode} constitutes a good base for future extensions, so it can be extended on many ways. We have used design patterns such as \textit{Abstract Mapper} or \textit{AngularJS} framework to obtain a maintainable code, making it easier to future developers to extend their project. Although the actual version of \textit{DuoCode} is stable, a lot of new features can be implemented:

\begin{itemize}
\item Integrate \textit{Open Badges}: a new online standard to recognize and verify learning. More info can be found at \url{http://openbadges.org/}

\item Show which parts of the code are wrong.

\item Intercalate questions in different languages.

\item Group languages with same programming paradigm.

\item Highlight syntax in real time.

\item New designs.

\item Improve the mobile version.

\item Implement database normalization.

\end{itemize}