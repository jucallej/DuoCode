%!TEX root = memoria_duocode_interfaz.tex

In this project we have implemented a REST web service and a \emph{front-end} which uses it, for creating a collaborative learning project of programming languages called DuoCode. This project has allowed the students that developed it to go in depth into technologies and paradigms that are not taught during the degree or that are merely referenced; additionally, it has allowed them to reinforce other knowledge about programming and planning projects. In the rest of the section we will introduce in the state of the art, the technological influences and our objectives.

\subsection{State of the art\label{subsec:state_art}}

Nowadays there are numerous ways of learning online. First, there are typical resources where learners read and perform few interactions. The leader per excellence in this modality is Wikipedia \cite{wiki}, an open and collaborative multi-language encyclopedia. To assist in solving specific doubts there are sites for questions and answers like Stack Overflow~\cite{stack}, were programmers help each other to solve particular problems. Additionally, many universities use \emph{virtual learning environments} like Moodle \cite{moodle}, which make easier the full management of subjects. There are also online courses, called MOOC (Massive Open Online Course) where the users can learn different subjects. These require material like videos with detailed explanations to be uploaded. Usually the purpose of these courses is to obtain some sort of diploma or certificate. Examples of this approach are Coursera~\cite{coursera}, EdX~\cite{edX}, and Miriada X~\cite{miriadaX}.

Another learning approach, specially when you want to expand something you already have a base of, is by solving exercises. \emph{Online judges} or \emph{automated assessment system} have been used for many years. There is a general analysis about the topic, where the viability of a collaborative project for teaching programming languages is studied~\cite{pimcdDuoCode14}. It explains that normally this type of systems, applied to programming languages, is based on the execution of several test cases to check the correction of the user's programs. The disadvantage of these systems is that they require the trainers to develop these tests in advance, a tough task. Besides, they only point out whether there is an error or not but they do not show where the mistake is.

Besides the world of programming there are other examples, such as Duolingo \cite{duolingo} which allows users to learn a language based on others that they already know. Its most relevant features are:

\begin{itemize}
\item
It makes learning similar to a game, where users earn points and experience as they progress. Furthermore it includes lives, which encourage users to focus their attention on the current task to avoid having to restart the level.

\item
It stores information regarding failures of the users to try to repeat these questions and force them to learn the concepts in a definitive manner.

\item
It includes timed practices and the possibility of certifying the level on the language through online tests.
\end{itemize}

There are other alternatives like Bussu \cite{bussu}, where users take simultaneously the roles of students and teachers and interact with each other in a social network with the objective of learning another language.

\subsection{Technological influences\label{subsec:influences}}
Before defining how to carry out this project, we wanted to plan from a high-level perspective which type of system to develop. All the subjects that are about to be named are the ones corresponding the University curriculum \cite{plan, boe}. 

During our degree we have learned, among other things, programming languages to write executable software for desktop computers, in subjects such as Fundamentals of Programming or Computer Programming Technology. We chose not to follow this approach, even though it could be appropriate for our project. We were certain that we could come up with a solution that would be much faster to test and use for the first time, without the need to install heavy software.

In other subjects such as Enterprise Software we learned to build and configure a CMS (Content Management System) and build a web page with it. This approach introduces more advantages, since it allows anybody with a web browser to access our platform. However, current CMSs do not allow to do such specific things as we would like. Besides, we would leave unused most of its features and we would not use them up to their full potential.

We also took web applications, where we learned to develop a web from scratch. We were interested in carrying out a development of this kind, since it would allow a high degree of flexibility, and potential users could use the platform quick and easily. Furthermore, the use of asynchronous calls can allow us to make a fully interactive and fluid web page to user interactions. For these reasons we decided to use HTML, CSS, and JavaScript in our project. In this subject we rely on knowledge gain on others such as Databases and Advanced Databases, where we learned to design and implement relational databases and use them from applications. Their use allows to establish interconnections (relations) between data (stored in different tables), and and whose main advantages are avoiding data duplicity and securing the integrity of related data. With this knowledge we can make fully interactive websites that stores a wide variety of data. For all of these facts we decided to use this kind of database.  

We also had a desire for making the content mobile friendly, either by adapting the website or or by making a mobile application. Some of us were enrolled on the subject Application Programming for Mobile Devices, and we saw an opportunity for showing what we could achieve.  

Lastly, we researched the utility of implementing a web service, a technology that uses a set of protocols and standards to exchange data between applications. The more similar approach that we had studied is the use of sockets in Operating Systems and Network Extension. Their use allows to abstract and split data manipulation from the graphical user interface, been able of even having multiple interfaces for the same application. For all of these we decided to follow this approach. We researched several ways of implementing a web service:

\begin{itemize}
\item
RPC (Remote Procedure Call) a protocol that allows one computer to execute code in another one in a remote location. We saw that there a several ways of implementing RPC, using different specifications, been SOAP one of the most popular.

\item
SOAP (Simple Object Access Protocol) a protocol that defines how different computers can compunicate by exchanging XML messages. Their principal disadvantages are that it relies on a service descriptor, that you have to update with every new functionality. Moreover, it uses XML, a markup language that has lost popularity facing the more light JSON. Lastly, not all programming languages offer utilities for integrating this kind of Web service.

\item
REST (Representational State Transfer) a stateless client/server protocol that uses HTTP and its methods (GET, POST, PUT, DELETE, ...) to check and modify the different resources. Frequently these are also called RESTful. They allow a lot of freedom when developing and, as programs only need to make HTTP requests to consume these web services, it is widely supported by numerous programming languages. Furthermore, they allow to present the data in JSON format. For these reasons these we decided to use this technology for our project.

\end{itemize}

\subsection{Objectives and proposals\label{subsec:proposals}}

Through the previous research we grounded our ideas. We decided to do a project for learning programming languages, similar to Duolingo. As it will be oriented to programming code, we decided to call it DuoCode, since you will learn new programming language from others that you already know.

The main features that we want to make available  to the user are:

\begin{itemize}
\item
Select the programming language that they already know and the one you want to learn. The system will show code in the language that the user already masters and will ask for the translation into the one the user wants to learn.

\item
The code will have a clear organization by subjects.

\item
Users will have a score that will rise as they progress. Furthermore, exercises will have a set of lives to be completed. This will favor the user's loyalty throw gamification.

\item
When users fail an exercise but they do not agree, they will be able to propose it as a solution. With the help of the community and moderators, the system will be able to add solutions in a specific programming language. 

\item
Exercises can be stored as favorites to check them again.

\item
There will be interaction with social networks. The user will be able to authenticate using some of them and share the results on them. 

\end{itemize}
Based on the conclusions extracted from the previous section, our objectives for the project are:

\begin{itemize}
\item
To develop a REST Web service for learning programming languages. We intend to follow these steps:

\begin{itemize}
\item
Specify the different REST resources and what information they receive and produce for each of the HTTP methods (GET, POST, PUT, and DELETE).

\item
Design a relational database to support the service, specifying the different tables and their relations.

\item
Implement and test the REST service.

\end{itemize}

\item
Develop at least one graphic interface using the service described above:

\begin{itemize}
\item
Design the interface using HTML/CSS, using test data.

\item
Connect the interface with the web service through asynchronous calls and JavaScript, using the real data provided by the REST service. The use of this technique is often called AJAJ (Asynchronous JavaScript And JSON)

\item
Develop an application for Android, or depending on the time constrains adapt the website for its visualization on mobile devices.

\end{itemize}

\end{itemize}

The rest of the memory is structured as follows: in Section~\ref{sec:s_web} we introduce our web service, while in Section~\ref{sec:front} the front-end that uses it is detailed.
Section~\ref{sec:req} presents the requirements and the database used on the project. Section~\ref{sec:manual} introduces the user's manual and Section~\ref{sec:conc} concludes and presents some lines of the future work.
Lastly, Appendix~\ref{app:req} describes in detail the requirements and the appendix~\ref{app:manual} the installation guide.